
\documentclass[a4paper,UKenglish,cleveref, autoref, thm-restate, numberwithinsect]{lipics-v2021}
%This is a template for producing LIPIcs articles.
%See lipics-v2021-authors-guidelines.pdf for further information.
%for A4 paper format use option "a4paper", for US-letter use option "letterpaper"
%for british hyphenation rules use option "UKenglish", for american hyphenation rules use option "USenglish"
%for section-numbered lemmas etc., use "numberwithinsect"
%for enabling cleveref support, use "cleveref"
%for enabling autoref support, use "autoref"
%for anonymousing the authors (e.g. for double-blind review), add "anonymous"
%for enabling thm-restate support, use "thm-restate"
%for enabling a two-column layout for the author/affilation part (only applicable for > 6 authors), use "authorcolumns"
%for producing a PDF according the PDF/A standard, add "pdfa"

%\pdfoutput=1 %uncomment to ensure pdflatex processing (mandatatory e.g. to submit to arXiv)
%\hideLIPIcs  %uncomment to remove references to LIPIcs series (logo, DOI, ...), e.g. when preparing a pre-final version to be uploaded to arXiv or another public repository

%\graphicspath{{./graphics/}}%helpful if your graphic files are in another directory

\usepackage[dvipsnames]{xcolor}
% Needed for tikz
\usepackage{tikzit}
\usepackage{mathtools}

\input{style1.tikzstyles}

\bibliographystyle{plainurl}% the mandatory bibstyle

\title{On the Separation of Conditional XPath \and
    \Large M.Sc. Thesis}

\titlerunning{Master's Thesis}

\author{V.R. Sathiyanarayana}{
    Chennai Mathematical Institute
    \and Roll number: \textsf{MCS202018}
    \and Advisors: Prof. C. Aiswarya and Prof. Paul Gastin
    \and \url{http://sathiyavrs.netlify.app}}{sathiyanarayana@cmi.ac.in}{}{}%TODO mandatory, please use full name; only 1 author per \author macro; first two parameters are mandatory, other parameters can be empty. Please provide at least the name of the affiliation and the country. The full address is optional. Use additional curly braces to indicate the correct name splitting when the last name consists of multiple name parts.

%\author{Joan R. Public\footnote{Optional footnote, e.g. to mark corresponding author}}{Department of Informatics, Dummy College, [optional: Address], Country}{joanrpublic@dummycollege.org}{[orcid]}{[funding]}

\authorrunning{V.R. Sathiyanarayana} % mandatory. First: Use abbreviated first/middle names. Second (only in severe cases): Use first author plus 'et al.'

\Copyright{V. R. Sathiyanarayana} %TODO: mandatory, please use full first names. LIPIcs license is "CC-BY";  http://creativecommons.org/licenses/by/3.0/ % TODO: Fill something here

\begin{CCSXML}
<ccs2012>
<concept>
<concept_id>10003752.10003790.10003793</concept_id>
<concept_desc>Theory of computation~Modal and temporal logics</concept_desc>
<concept_significance>500</concept_significance>
</concept>
</ccs2012>
\end{CCSXML}

\ccsdesc[500]{Theory of computation~Modal and temporal logics}

%\ccsdesc[100]{\textcolor{red}{Replace ccsdesc macro with valid one}} %TODO mandatory: Please choose ACM 2012 classifications from https://dl.acm.org/ccs/ccs_flat.cfm

\keywords{Logic, Modal and temporal logic} %TODO mandatory; please add comma-separated list of keywords

\category{} %optional, e.g. invited paper

\relatedversion{} %optional, e.g. full version hosted on arXiv, HAL, or other respository/website
%\relatedversiondetails[linktext={opt. text shown instead of the URL}, cite=DBLP:books/mk/GrayR93]{Classification (e.g. Full Version, Extended Version, Previous Version}{URL to related version} %linktext and cite are optional

%\supplement{}%optional, e.g. related research data, source code, ... hosted on a repository like zenodo, figshare, GitHub, ...
%\supplementdetails[linktext={opt. text shown instead of the URL}, cite=DBLP:books/mk/GrayR93, subcategory={Description, Subcategory}, swhid={Software Heritage Identifier}]{General Classification (e.g. Software, Dataset, Model, ...)}{URL to related version} %linktext, cite, and subcategory are optional

%\funding{(Optional) general funding statement \dots}%optional, to capture a funding statement, which applies to all authors. Please enter author specific funding statements as fifth argument of the \author macro.

\acknowledgements{I want to thank my advisors \href{https://www.cmi.ac.in/~aiswarya/}{Prof. C. Aiswarya} and \href{http://www.lsv.fr/~gastin/}{Prof. Paul Gastin} for the incredible amount of advice, encouragement, and direction they provided during the difficult parts of this endeavour.}%optional

\nolinenumbers %uncomment to disable line numbering

%Editor-only macros:: begin (do not touch as author)%%%%%%%%%%%%%%%%%%%%%%%%%%%%%%%%%%
% \EventEditors{C. Aiswarya}
% \EventNoEds{2}
% \EventLongTitle{Master's Thesis}
% \EventShortTitle{MSc. (CS)}
% \EventAcronym{MSc.}
% \EventYear{2022}
% \EventDate{June 4--7, 2022}
% \EventLocation{Chennai, India}
% \EventLogo{}
% \SeriesVolume{1}
\ArticleNo{38}
%%%%%%%%%%%%%%%%%%%%%%%%%%%%%%%%%%%%%%%%%%%%%%%%%%%%%%

%% A few shortcuts for Leftarrow, etc
\def\Larrow{\ensuremath\mathord{\Leftarrow}}
\def\Rarrow{\ensuremath\mathord{\Rightarrow}}
\def\Uarrow{\ensuremath\mathord{\Uparrow}}
\def\Darrow{\ensuremath\mathord{\Downarrow}}

\newcommand{\myvec}[1]{\overrightarrow{\mathbf{#1}}}

\begin{document}

\maketitle

%TODO mandatory: add short abstract of the document
\begin{abstract}
    Separation was introduced by Dov Gabbay for the $S, U$ temporal language over linear time and was shown to imply its expressive equivalence with the first-order monadic logic of order. The effectiveness and flexibility of Gabbay's arguments led to the practice of characterizing the expressive power of new temporal logics by attempting to separate them. Accordingly, Maarten Marx proposed a separation property for Conditional XPath, an extension of the XML query language XPath. Conditional XPath can be viewed as a temporal logic over ordered trees, and has been proven to be expressively equivalent to the corresponding first-order language. A mistake in Marx's proof of separation was later discovered, and consequent attempts to fix this mistake have only produced negative results.

    In this thesis, we explore some implications of Marx's work. We show that his arguments can be used to separate a subclass of all formulas. We then describe EF games designed for Conditional XPath, and use these games to characterize the difficulty of separating a particular formula outside this class. We conjecture that this formula cannot be separated. Separately, we justify Marx's partitioning of ordered trees by unexpectedly deriving them.
\end{abstract}

\section{Introduction}
\label{sec:introduction}

Temporal logics are excellent languages for making statements about systems that change over time. A variety of these logics have been studied; some differ in the way they model time, some in the way they interpret systems that change with time, and some in the mechanisms they provide to reason through time. A popular example is the Linear Temporal Logic, where time is modelled as a linear order with a clear beginning, and systems that change with time are tracked using classical propositional logic. Other logics have modelled branching time, concurrent systems, and other complex graph structures.

These logics have found applications in many domains, from the formal verification of the behaviour of computer programs (see \cite{vardiLTL}) to database management systems (see \cite{gabbay1994}) and even to planning problems in Artificial Intelligence (see \cite{planningLTL}). A part of the reason for the ubiquity of temporal logics is the combination of their attractive expressiveness and complexity properties. The satisfiability of linear-temporal logic, for example, is PSPACE-complete (see \cite{vardiLTL}), which is a significant improvement from the non-elementary complexity of the same problem in the similarly expressive first-order monadic logic of order.

The separation property, invented by Dov Gabbay in \cite{Gabbay1981}, is a strangely influential consequence of the design of popular temporal languages. Simply put, it requires all formulas in the language to be equivalent to a variant made up of formulas purely concerned with certain \textit{regions} of time. Surprisingly, this property is linked to expressive completeness: a minimally expressive temporal logic with the separation property can express any first-order specification. In \cref{sec:linear-time}, we will detail the separation property over linear time and how it implies functional completeness.

% Separation has many interesting applications beyond expressive completeness. Gabbay describes one fascinating use-case in \cite{DecPastImpFuture89}. Here, he notes that while general formulas in the temporal logic of $U$ and $S$ (which we describe in \cref{sec:preliminaries}) are \textit{declarative} (i.e., they \textit{declare} correct behaviour), formulas purely concerned with the future can be viewed as instructions (to produce correct behaviour) in an imperative programming language. This, in addition to the separation property, allows us to write every statement in the logic as a conjunction of requirements that (1) inspect the past to see if it satisfies a declarative property, and if the past is correct (2) require future behaviour to be of a certain form. % NOTE: Unfortunately, I probably shouldn't describe this paper entirely here.

Separation has many interesting applications beyond expressive completeness. A beautiful one can be found in \cite{DecPastImpFuture89}, which describes how a separable temporal language can simultaneously be \textit{declarative} (i.e., specifies correct behaviour) and \textit{imperative} (i.e., provides instructions to achieve correct behaviour). Another use-case, described in \cite{Lichtenstein1985TheGO}, shows that formulas in a separable logic can be written as a boolean combination of safety and liveness properties. A more thorough exposition of the applications of separation can be found in \cite{GabbayBirthday05}.

As we earlier implied, temporal languages have been studied for a variety of models of time. A particularly interesting and useful class of temporal logics model time as \textit{unranked ordered trees}. The usefulness of these languages stems directly from the fact that many natural structures in computer science (the nested words of \cite{alur2009}, XML, and other structured data) can be encoded as ordered trees.

In \cite{xpathConditional}, Marx introduces \textit{Conditional XPath}, an extension of the XML path specification language XPath. XPath statements are evaluated over the well-understood Document Object Model of the input XML file, which arranges the contents of the file in an ordered tree. Marx motivates the extensions contained in Conditional XPath by showing that the base XPath language isn't sufficiently expressive (see \cite{xpathConditional}). By contrast, Conditional XPath is capable of expressing any first-order property over finite ordered trees (see \cite{marx2005conditional}). Marx simplifies Conditional XPath to its core in \cite{xpathComplete}, and proposes a separation property for it.

Unfortunately, the proof Marx employs for his separation property is incorrect. The mistake lies in an unnamed lemma, and has been commented on in \cite{BeJe07, nwtl}. Further attempts to prove a separation result have been made in \cite{BeCl16}, with mostly negative results.

In this thesis, we explore the separation of Conditional XPath over unranked ordered trees. We note Marx's mistakes, and show that his arguments in \cite{xpathComplete} naturally lead to a \textit{partial} separation property, whereby a subset of formulas can be separated. We also believe that certain simple formulas \textit{cannot} be separated; through the use of EF games adapted from \cite{EtWi00}, we show a result that indicates that separation of a particular simple formula is unlikely. Separately, we also justify Marx's choice of regions by showing that a more desirable set of regions cannot yield separation.

The structure of this document is as follows. In \cref{sec:preliminaries}, we discuss basic notions regarding temporal languages. In \cref{sec:linear-time}, we provide a (mostly) self-contained exposition of Gabbay's proof of separation over linear time (see \cite{DecPastImpFuture89, gabbay1994}). We discuss our main results in \cref{sec:ordered-trees}, and present avenues for future work in \cref{sec:conclusions}.

\section{Preliminaries}
\label{sec:preliminaries}

Before discussing separation, we need to define some standard notions. A \textit{flow of time} is simply a non-empty set $T$ partially ordered by the binary relation $<$. We symbolically refer to these flows by the pair $(T, <)$. Examples include $(\mathbb{N}, <)$ and $(\mathbb{R}, <)$ with their natural ordering, unordered trees with the descendant relation, and Mazurkiewicz traces. We will consider the truth values of propositions (from a fixed set $\mathcal{P}$) at points on these flows.

The first-order vocabulary over these structures contains the ordering relation $<$ and a collection of \textit{monadic} relations $Q_1, Q_2, \cdots$ that match the propositions $q_1, q_2, \cdots$ in $\mathcal{P}$. An assignment $h$ of atoms in a time flow $(T, <)$ assigns to each $Q_i$ a subset of $T$ where the atom $q_i$ is true. Augmented with the assignment, the triplet $(T, <, h)$ is called a \textit{temporal structure}. First-order formulas are evaluated over these structures in the usual way. In this discussion, we pay special attention to first-order formulas with a single free-variable; they quite naturally mirror temporal formulas.

Instead of variables and quantification, temporal languages employ \textit{temporal connectives} to reason through time. Popular connectives used in temporal languages over linear time include $F$, $P$, $G$, $H$, $U$, and $S$, each called \textit{future}, \textit{past}, \textit{globally}, \textit{history}, \textit{until} and \textit{since} respectively. In this paper, we will limit our discussion to connectives that are definable by monadic first-order formulas.
\begin{note*}
    Unless otherwise specified, when we refer to \textit{connectives} in this thesis, we mean \textit{temporal connectives}.
\end{note*}

Temporal formulas are evaluated at points in time. At a point $t \in T$ in the temporal structure $\mathcal{M} = (T, <, h)$, the atom $q \in \mathcal{P}$ is evaluated as
\begin{equation*}
    \mathcal{M}, t \vDash q \Longleftrightarrow (T, <, h[x \mapsto t]) \vDash Q(x) \Longleftrightarrow t \in h(Q)\\
\end{equation*}
As per the standard notation, the assignment $h[x \mapsto t]$ assigns the time point $t$ to the first-order variable $x$. To simplify the presentation, we use $\mathcal{M}, t \vDash \varphi(t)$ to mean $(T, <, h[x \mapsto t]) \vDash \varphi(x)$.

For a generic connective $\sharp$ of arity $n$, let $\varphi_\sharp(t, X_1, \cdots X_n)$ be the monadic first-order formula defining it. Here, $t$ is the point in time that the connective is evaluated against and the $X_i$ are monadic (second-order) variables. These variables expect a single-variable first-order formula, as shown below
\begin{equation*}
    \mathcal{M}, t \vDash \sharp(A_1, \cdots A_n) \Longleftrightarrow \mathcal{M}, t \vDash \varphi_\sharp(t, \alpha_{A_1}, \cdots \alpha_{A_n})
\end{equation*}
Here, $A_i$ are temporal formulas and $\alpha_{A_i}$ are their first-order translations. Notably, $\varphi_\sharp$ can only quantify over elements in the domain $T$; it cannot use second order quantifiers.

We illustrate this behaviour with an example. The connective $F$ is defined by the formula
\begin{equation*}
    \varphi_F(t, X) \triangleq \exists x.\, (t < x) \land X(x)
\end{equation*}
Hence, we have
\begin{equation*}
    \mathcal{M}, t \vDash F(q_i) \Longleftrightarrow \varphi_F(t, Q_i)
\end{equation*}
We similarly define the other main connectives
\begin{equation*}
    \begin{aligned}
        \varphi_P(t, X) &\triangleq \exists x.\, (x < t) \land X(x)\\
        \varphi_G(t, X) &\triangleq \forall x.\, (t < x) \land X(x)\\
        \varphi_H(t, X) &\triangleq \forall x.\, (x < t) \land X(x)\\
        \varphi_U(t, X_1, X_2) &\triangleq \exists x.\, \left[ \left( t < x \right) \land X_1(x) \land \forall y \left( \left( t < y < x \right) \to X_2(y) \right) \right]\\
        \varphi_S(t, X_1, X_2) &\triangleq \exists x.\, \left[ \left( x < t \right) \land X_1(x) \land \forall y \left( \left( x < y < t \right) \to X_2(y) \right) \right]\\
    \end{aligned}
\end{equation*}
This gives us the following
\begin{equation*}
    \mathcal{M}, t \vDash P(F(q_i)) \Longleftrightarrow \mathcal{M}, t \vDash \varphi_P(t, \varphi_F[X \mapsto Q_i])
\end{equation*}
This is a simple composition of the definitions of $\varphi_F$ and $\varphi_P$.

Note that, unlike the typical definition of $U$, $\varphi_U$ doesn’t rely on the present point $t$. Such an until is referred to in the literature by either the \textit{strict} until (see \cite{gastinStrictUntil06}) or the \textit{strong} until (see \cite{BeCl16}). This particular behaviour makes observing separation much easier.

We now define the notion of \textit{Expressive Completeness}.
\begin{definition}[Expressive Completeness]
    \label{expressive-completeness-definition}
    A temporal language is \textbf{first-order expressively complete} over a class of time flows if there exists a temporal formula $A$ for any first-order formula with one free variable $\varphi(t)$ such that
    \begin{equation*}
        \mathcal{M}, t \vDash A \Longleftrightarrow \mathcal{M}[x \mapsto t] \vDash \varphi(x)
    \end{equation*}
    for any flow $\mathcal{M}$ in the class.
\end{definition}
Expressive completeness is a natural milestone for temporal languages, and is the primary application of separation. As promised, we will study this in detail in \cref{sec:linear-time}. Separately, note that a flow of time $(T, <)$ is termed to be expressively complete if there exists an expressively complete temporal language over it.

\section{Linear Flows}
\label{sec:linear-time}

In \cite{gabbay1994}, Gabbay showed how the temporal language $\mathbf{L}$ with the \textit{strict until} $U$ and \textit{strict since} $S$ connectives satisfies the separation property over the integer time flow $(\mathbb{Z}, <)$.

To discuss this further, we need the notion of \textit{regions} and \textit{pure formulas}. Informally, the flow of time $(T, <)$ is partitioned into a set of regions. The exact partition and the area covered by the individual regions depends on the position of the time point $t$ where the temporal formula is being evaluated. For the flow $(\mathbb{Z}, <)$ and the point $t \in \mathbb{Z}$, Gabbay selected three regions:
\begin{itemize}
    \item The \textit{past} of $t$, formally defined as $\{ x \in \mathbb{Z} \mid x < t\}$.
    \item The \textit{present}, which is simply $\{ t \}$.
    \item The \textit{future} of $t$, which naturally is $\{x \in \mathbb{Z} \mid t < x\}$
\end{itemize}
Note that these regions are disjoint, and that the union of these regions produces the entire flow. Also, notice that these regions are first-order definable.

\begin{figure}
    \centering
    \tikzfig{separation-linear}
    \caption[]{\emph{Regions for linear separation.} The present is black, the past is \textcolor{OliveGreen}{green}, and the future is \textcolor{blue}{blue}.}
    \label{fig:linear-separation-regions}
\end{figure}

% TODO: Consider writing these in \begin{definition}
Now, we define \textit{pure formulas}. For any flow $(T, <)$, we call two assignments $h$ and $h'$ to be in \textit{agreement} over a region $R \subset T$ \textit{iff} for every atom $q \in \mathcal{P}$ and every point $s \in R$,
\begin{equation*}
    s \in h(Q) \Longleftrightarrow s \in h'(Q)
\end{equation*}
Now, call a temporal formula $\gamma$ \textit{pure} with respect to a region $R$ if, for any two assignments $h$ and $h'$ that agree on $R$,
\begin{equation*}
    (T, <, h), t \vDash \gamma \Longleftrightarrow (T, <, h'), t \vDash \gamma
\end{equation*}
In other words, $\gamma$ is true on $h'$ \textit{iff} $\gamma$ is true on $h$. We use the terms \textit{pure past}, \textit{pure present}, or \textit{pure future} to denote pure formulas in the past, present, and future regions respectively.

It's easy to see that formulas that don't use the $S$ and $U$ connectives are pure-present. In a similar vein, formulas that are rooted by a $S$ connective and don't use the $U$ connective are pure-past. We formalize this understanding using the notion of \textit{syntactically pure} formulas. To simplify presentation, we refer to formulas rooted by a $U$ (or an $S$) as $U$-formulas (or $S$-formulas, respectively).
\begin{definition}
    A temporal formula $\varphi$ in the temporal language of $S$ and $U$ is
    \begin{enumerate}
        \item syntactically pure-present iff it doesn't use the $S$ and $U$ connectives.
        \item syntactically pure-past iff it is a boolean combination of $S$-formulas that don't use the $U$ connective.
        \item syntactically pure-future iff it is a boolean combination of $U$-formulas that don't use the $S$ connective.
    \end{enumerate}
\end{definition}

Finally, call a formula $\gamma$ \textbf{syntactically separated} if it is a boolean combination of syntactically pure formulas. Now, we can state the separation property.
\begin{claim*}[Separation Property for linear flows]
    \label{claim:separation-property-linear-time}
    Every temporal formula $A$ in the language of $S$ and $U$ over linear time can be equivalently represented by a separated formula.
\end{claim*}
The proof of this claim is quite involved, and is presented in full detail in \cite{gabbay1994, DecPastImpFuture89}. In the next few sections, we'll provide most of Gabbay et. al.'s proof. Parts of this proof will be later used in \cref{sec:ordered-trees}. To mirror their notation, we'll write $U$ formulas as $U(p, q)$ instead of $q \,\mathcal{U}\, p$.

\subsection{Separating $S$ and $U$ over linear time}

As a reminder, we restate the definitions of $U$ and $S$
\begin{equation*}
    \begin{aligned}
        \mathcal{M}, t \vDash U(p, q) &\Longleftrightarrow \mathcal{M}, t \vDash \exists x.\, (t < x) \land P(x) \land \forall y \left( t < y < x \to Q\left( y \right) \right)\\
        \mathcal{M}, t \vDash S(p, q) &\Longleftrightarrow \mathcal{M}, t \vDash \exists x.\, (x < t) \land P(x) \land \forall y \left( x < y < t \to Q\left( y \right) \right)
    \end{aligned}
\end{equation*}
For convenience, we refer to the left condition ($p$) in $U(p, q)$ as the \textit{target} condition and the right condition ($q$) as the \textit{path} condition. Observe that, over linear time, a formula that only uses the $U$ connective is a pure future formula and a formula that only uses the $S$ connective is a pure past formula. The task, therefore, is to transform formulas that use both $U$ and $S$.

Over the integer time flow $(\mathbb{Z}, <)$, these connectives naturally possess the following properties
\begin{equation}
    \label{eq:or-and-S-U}
    \begin{aligned}
        U(\alpha \lor \beta, \gamma) &\equiv U(\alpha, \gamma) \lor U(\beta, \gamma)\\
        U(\alpha, \beta \land \gamma) &\equiv U(\alpha, \beta) \land U(\alpha, \gamma)
    \end{aligned}
\end{equation}
We will refer to this property later.

In addition, their negations can be usefully rewritten as
\begin{equation*}
    \begin{aligned}
        \lnot U(\alpha, \beta) &\equiv G(\lnot \alpha) \lor U(\lnot \alpha \land \lnot \beta, \lnot \alpha) \\
        \lnot S(\alpha, \beta) &\equiv H(\lnot \alpha) \lor S(\lnot \alpha \land \lnot \beta, \lnot \alpha)
    \end{aligned}
\end{equation*}
where the semantics of $G$ and $H$ are
\begin{equation*}
    \begin{aligned}
        \mathcal{M}, t \vDash G(\alpha) &\Longleftrightarrow \mathcal{M}, t \vDash \forall t'.\; [(t' > t) \to \varphi_\alpha(t')] \\
        \mathcal{M}, t \vDash H(\alpha) &\Longleftrightarrow \mathcal{M}, t \vDash \forall t'.\; [(t' < t) \to \varphi_\alpha(t')]
    \end{aligned}
\end{equation*}
Here, $\varphi_\alpha$ is the first-order translation of $\alpha$.

Our strategy involves \textit{pulling-out} $U$s from inside an $S$ and vice versa. We accomplish this by writing all temporal formulas in a standard notation and strategically applying a sequence of \textit{elimination} rules. In the next section, we describe these rules.

\subsubsection{Eliminations}
\label{sec:eliminations-linear}

Let $\alpha$, $\beta$, $\varphi$ and $\psi$ be boolean combinations of propositional atoms. In the following subsections, we pull out a $U(\varphi, \psi)$ from inside an $S$ under a variety of minimal configurations.

\paragraph*{$\blacktriangleright\;\, S(\alpha \land U(\varphi, \psi), \beta)$}
This formula requires $\alpha \land U(\varphi, \psi)$ to be true at a point $t'$ in the past of $t$. This in turn implies $\varphi$ at some point $t''$ ahead of $t'$. This naturally breaks down into three cases: $t'' > t$, $t'' = t$, and $t' < t'' < t$. The translation is
\begin{equation*}
    \begin{aligned}
        &S(\varphi \land \beta \land S(\alpha, \psi \land \beta), \beta)\\
        \lor \quad &\left(S(\alpha, \psi \land \beta) \land \left(\varphi \lor \left(\psi \land U(\varphi, \psi) \right) \right) \right)
    \end{aligned}
\end{equation*} \lipicsEnd

\paragraph*{$\blacktriangleright\;\, S(\alpha \land \lnot U(\varphi, \psi), \beta)$}
In this case, we immediately rewrite $\lnot U(\varphi, \psi)$ as $G(\lnot \alpha) \lor U(\lnot \alpha \land \lnot \beta, \lnot \alpha)$. This gives us
\begin{equation*}
    \begin{aligned}
        &S(\alpha \land G(\lnot \varphi), \beta) \\
        \lor \quad &S(\alpha \land U(\lnot \varphi \land \lnot \psi, \lnot \varphi), \beta)
    \end{aligned}
\end{equation*}
where each disjunct can be translated by applying the ideas used to rewrite $S(\alpha \land U(\varphi, \psi), \beta)$. \lipicsEnd

\paragraph*{$\blacktriangleright\;\, S(\alpha, U(\varphi, \psi))$}
It's instructive to understand how $S(\alpha, U(\varphi, \psi))$ could be translated. Unlike the previous cases, $U(\varphi, \psi)$ fragment needs to be true at each point in the path to $\alpha$. This could involve multiple segments in this path where $\psi$ is true till $\varphi$ is true. Wonderfully, this is \textit{indistinguishable} from the case where, at each point in the path, either $\varphi$ or $\psi$ is true. This formula is translated to % TODO: Fix this formula
\begin{equation*}
    \begin{aligned}
        &S(\alpha, \bot) \\
        \lor \quad &S(\alpha, \varphi \lor \psi) \land \left[ \varphi \lor \left(\psi \land U\left(\varphi, \psi \right) \right) \right] \\
    \end{aligned}
\end{equation*}
Here, $S(\alpha, \bot)$ can only be true if $\alpha$ is true at the previous point. Otherwise, we'll need $U(\varphi, \psi)$ to be satisfied at the previous location, hence the $\varphi \lor (\psi \land U(\varphi, \psi))$ at the present. At each point $t'$ in the path to $\alpha$, if $t' + 1 \vDash \varphi$, $t' \vDash U(\varphi, \psi)$. Otherwise, $t' + 1 \vDash \psi$. At this point, we can use an inductive argument, starting from the previous point, to prove the correctness of this translation. \lipicsEnd

\paragraph*{$\blacktriangleright\;\, S(\alpha, \beta \lor U(\varphi, \psi))$}

% I had some trouble extending this analysis to translate $S(\alpha, \beta \lor U(\varphi, \psi))$. Gabbay's presentation is confusing, as he provides an unnecessarily complicated translation for this, but later presents a cleverer translation embedded in his translation of $S(\alpha \land U(\varphi, \psi), \beta \lor U(\varphi, \psi))$, a more complex formula. Gratifyingly, this translation uses elements of my translation of $S(\alpha, U(\varphi, \psi))$.

The idea is to attempt to enforce $U(\varphi, \psi)$ at each point in the path to $\alpha$ \textit{only if} it is essential. This harsh necessity can be estimated by looking a \textit{little further} down the path for a point where $\lnot \beta$ was true (implying the requirement for $U(\varphi, \psi)$ to be true at that point) with $\lnot \varphi$ true at all points on the ``little'' path to it. Call such a point an \textit{unfulfilled} point. The formula $S(\lnot \beta \land \lnot \alpha, \lnot \varphi \land \lnot \alpha)$ detects these unfulfilled points. Here, the $\lnot \alpha$ ensures that we specifically look for points in the future of $\alpha$, the leftmost point in our consideration.

Observe that, for $U(\varphi, \psi)$ to be true at a detected unfulfilled point, the current point must either satisfy $\varphi$, or satisfy $\psi$ and let a future point satisfy $\varphi$. This suggests the following quick fix:
\begin{equation*}
    S(\lnot \beta, \lnot \varphi \land \lnot \alpha) \to (\varphi \lor \psi)
\end{equation*}

It's important to recognize that we are capable of performing this fix at each point on the path to $\alpha$. If we implemented this idea, and if we recognized and ``fixed'' an unfulfilled point that's 3 steps away, we would've done the same 2 steps away and 1 step away. If $\varphi$ was true at the second step, we wouldn't have recognized this point three steps away (as it is no longer unfulfilled; it has been fulfilled at the second step). Hence, $\psi$ must've been true at the second step. This means that, until we see a $\varphi$ true at the current point, we will continue to ``fix'' the unfulfilled point. This crucial observation forms the backbone of the following translation.
\begin{equation*}
    \begin{aligned}
    S(\alpha, \lnot \alpha \land (S(\lnot \beta \land \lnot \alpha, \lnot \varphi \land \lnot \alpha) \to (\varphi \lor \psi))) \\
    \quad \land \quad S(\lnot \beta \land \lnot \alpha, \lnot \varphi \land \lnot \alpha) \to (\varphi \lor (\psi \land U(\varphi, \psi)))
    \end{aligned}
\end{equation*} \lipicsEnd

\paragraph*{$\blacktriangleright\;\, S(\alpha, \beta \lor \lnot U(\varphi, \psi))$}
This case is very similar to the previous case. The points we search for must be in danger of satisfying $U(\varphi, \psi)$; hence, we look for $S(\lnot \beta \land \lnot \alpha, \psi \land \lnot \alpha)$. We fix these points by requiring $\varphi$ to be false. In the worst-case, we've dragged on the possible \textit{until} to the present, at which point we extinguish all hope. This gives us the overall translation:
\begin{equation*}
    \begin{aligned}
    S(\alpha, \lnot \alpha \land (S(\lnot \beta \land \lnot \alpha, \psi \land \lnot \alpha) \to \lnot \varphi)) \\
    \quad \land \quad S(\lnot \beta \land \lnot \alpha, \psi \land \lnot \alpha) \to ((\lnot \psi \land \lnot \varphi) \lor (\lnot U(\varphi, \psi)))
    \end{aligned}
\end{equation*} \lipicsEnd

\paragraph*{$\blacktriangleright\;\, S(\alpha \land U(\varphi,\psi), \beta \lor U(\varphi, \psi))$}
This is a neat combination of $S(\alpha \land U(\varphi, \psi), \beta)$ and $S(\alpha, \beta \lor U(\varphi, \psi))$. The translation is simple. \footnote{\textit{I believe Gabbay made a typo in this particular example. \cite{xpathComplete} mentions this.}}
\begin{equation*}
    \begin{aligned}
        \quad S(\alpha, \psi) \land (\varphi \lor (\psi \land U(\varphi, \psi))) \\
        \lor \quad \quad S(\varphi \land S(\alpha, \psi), S(\lnot \beta, \lnot \varphi) \to \varphi \lor \psi)\\
        \quad \land \quad S(\lnot \beta, \lnot \varphi) \to (\varphi \lor (\psi \land U(\varphi, \psi)))
    \end{aligned}
\end{equation*} \lipicsEnd

We can similarly write separated equivalents for all possible $S(\alpha \land \pm U(\varphi, \psi), \beta \lor \pm U(\varphi, \psi))$. The details are presented in \cite{gabbay1994, DecPastImpFuture89}.

\subsubsection{Putting it all together}

The eliminations presented in the previous section lend credence to the idea of separation. Amazingly, Gabbay presents a neat induction scheme that builds on these rules to separate \textit{every} temporal formula in the language. In this section, we present an overview of his arguments (which are presented in greater detail in \cite{gabbay1994}).

\begin{lemma}
\label{lemma:separation-linear-step1}
    Let $\varphi$ and $\psi$ be pure-present formulas and $\alpha$ and $\beta$ be formulas such that the only appearance of a $U$ in either of them is $U(\varphi, \psi)$, and that $U$ isn't nested inside an $S$. Then $S(\alpha, \beta)$ can be written as a syntactically separated formula where the only appearance of $U$ is $U(\varphi, \psi)$.
\end{lemma}
\begin{proof}
    We start by writing $\alpha$ and $\beta$ in their disjunctive and conjunctive normal forms respectively. During this transformation, we treat all top-level instances of $U$ and $S$ in them as atomic propositions. This gives us
    \begin{equation*}
        \begin{aligned}
            \alpha &\equiv \bigvee_i \left( \alpha_{i, 1} \land \alpha_{i, 2} \land \cdots \land \alpha_{i, m_i} \right)\\
            \beta &\equiv \bigwedge_j \left( \beta_{j, 1} \lor \beta_{j, 2} \lor \cdots \lor \beta_{j, n_j} \right)
        \end{aligned}
    \end{equation*}
    Here, the literals $\alpha_{i, k}$ and $\beta_{i, k}$ are composed of propositional atoms, $S$ formulas, and $U(\varphi, \psi)$. We use the above and \cref{eq:or-and-S-U} to write $S(\alpha, \beta)$ as
    \begin{equation}
        \label{eq:dnf-form-S-alpha-beta}
        \begin{aligned}
            S(\alpha, \beta) &\longmapsto S\left(\bigvee_i (\alpha_{i, 1} \land \cdots \land \alpha_{i, m_i}), \beta\right)\\
            &\longmapsto \bigvee_i S(\alpha_{i, 1} \land \cdots \land \alpha_{i, m_i}, \beta)\\
            &\longmapsto \bigvee_i S\left(\alpha_{i, 1} \land \cdots \land \alpha_{i, m_i}, \bigwedge_j \left( \beta_{j, 1} \lor \cdots \lor \beta_{j, n_i} \right) \right)\\
            &\longmapsto \bigvee_i \bigwedge_j S\left(\alpha_{i, 1} \land \cdots \land \alpha_{i, m_i}, \beta_{j, 1} \lor \cdots \lor \beta_{j, n_i} \right)
        \end{aligned}
    \end{equation}
    In the resulting formula, the target of each top-level $S$ is a conjunction of literals, and the path condition is a disjunction of literals. Notably, if $U(\varphi, \psi)$ doesn't appear in the target and the path of a top-level $S$ formula, that subformula is a pure-past formula.

    Hence, we focus our attention on the top-level $S$ formulas containing $U(\varphi, \psi)$. In one such formula, let $\alpha'$ be the conjunction of all literals in the target that aren't $U(\varphi, \psi)$ or its negation. Similarly, let $\beta'$ be the disjunction of all literals in the path that aren't $U(\varphi, \psi)$ or its negation. This lets us write that formula as one of the following
    \begin{equation*}
        \begin{aligned}
            &S(\alpha' \land \pm U(\varphi, \psi), \beta')\\
            &S(\alpha', \beta' \lor \pm U(\varphi, \psi))\\
            &S(\alpha' \land \pm U(\varphi, \psi), \beta' \lor \pm U(\varphi, \psi))\\
        \end{aligned}
    \end{equation*}
    Clearly, the eliminations we explored in the previous section can separate this formula! Additionally, note that the only $U$ formula in the RHS of the eliminations is $U(\varphi, \psi)$, satisfying the condition specified in the beginning of the lemma.

    Applying these elimination rules to each top-level $S$ containing a $U(\varphi, \psi)$ produces a separated formula equivalent to $S(\varphi, \psi)$. This completes the proof.
\end{proof}

The second step of the induction scheme is to consider cases where $U(\varphi, \psi)$ is nested under multiple levels of $S$.
\begin{lemma}
\label{lemma:separation-linear-step2}
    Let $\varphi$ and $\psi$ be pure-present formulas, and let $\gamma$ be a formula such that the only appearance of a $U$ in $\gamma$ is $U(\varphi, \psi)$. Then, $\gamma$ can be written as a syntactically separated formula where the only appearance of a $U$ is $U(\varphi, \psi)$.
\end{lemma}
\begin{proof}
    We show this lemma by inducting on the pair $(n_1, n_2)$, where $n_1$ is the maximum number of nested $S$es above a $U(\varphi, \psi)$ and $n_2$ is the number of $U(\varphi, \psi)$ nested inside $n_1$ $S$es.
    \begin{description}
        \item[Base case.] Here, $n_1 = 0$, and $\gamma$ is already separated.
        \item[Induction step.] Pick the most deeply nested subformula $S(\alpha, \beta)$ of $\gamma$ such that all instances of $U(\varphi, \psi)$ in $\alpha$ and $\beta$ are not nested inside an $S$. Applying lemma \ref{lemma:separation-linear-step1} to $S(\alpha, \beta)$ strictly reduces $(n_1, n_2)$, allowing us to use the induction hypothesis. Remember, lemma \ref{lemma:separation-linear-step1} only generates formulas where then only appearance of $U$ is $U(\varphi, \psi)$, which is required to use the induction hypothesis. This completes the proof.
    \end{description}
\end{proof}
The next step generalizes this approach to different (basic) until subformulas.
\begin{lemma}
\label{lemma:separation-linear-step3}
    Let $\varphi_1, \varphi_2, \ldots \varphi_n$ and $\psi_1, \psi_2, \ldots \psi_n$ be pure present formulas and $\gamma$ be a formula such that all appearances of $U$ in $\gamma$ are of the form $U(\varphi_i, \psi_i)$ for some $i \in \{1, 2, \ldots n\}$. Then, $\gamma$ can be written as a syntactically separated formula.
\end{lemma}
\begin{proof}
    Predictably, we induct on $n$.
    \begin{description}
        \item[Base case.] This is $n = 1$, identical to lemma \ref{lemma:separation-linear-step2}.
        \item[Induction case.] Introduce new propositional atoms $p_1, p_2, \ldots p_{n-1}$. For each $i \in \{1, \ldots n - 1\}$, replace each occurrence of $U(\varphi_i, \psi_i)$ in $\gamma$ with $p_i$ to produce $\gamma'$. Notice that the only $U$ formula in $\gamma'$ is $U(\varphi_n, \psi_n)$. This allows the application of lemma \ref{lemma:separation-linear-step2} to $\gamma'$ to produce its separated equivalent, $\gamma''$.

        Observe that the only occurrence of $U$ in $\gamma''$ is $U(\varphi, \psi)$. Hence, the $p_i$ appear in the pure-present or the pure-past components of $\gamma''$. Replace each instance of $p_i$ in $\gamma''$ with $U(\varphi_i, \psi_i)$ to produce $\gamma'''$. After replacement, apply the induction hypothesis on each previously pure-past component of $\gamma'''$ to complete the proof.
    \end{description}
    \begin{remark*}
        It isn't difficult to see that we cannot use lemma \ref{lemma:separation-linear-step2} if we introduce a single atom $p_n$ to represent $U(\varphi_n, \psi_n)$. Introducing more atoms is essential to the overall induction structure.
    \end{remark*}
\end{proof}
We can now finally consider the case of nested $U$s.
\begin{lemma}
    \label{lemma:separation-linear-step4}
    Let $\gamma$ be a formula that doesn't contain $S$es nested inside a $U$. Then, $\gamma$ can be separated.
\end{lemma}
\begin{proof}
    We cleverly induct on the maximum nesting depths of $U$s under an $S$. Let $n$ be the maximum $U$-nesting depth of $\gamma$.
    \begin{description}
        \item[Base case.] This is $n = 1$, which is lemma \ref{lemma:separation-linear-step3}.
        \item[Induction step.] Suppose there are $m$ subformulas rooted at a $U$ that aren't under a $U$ and are under an $S$. Introduce $2m$ atoms $\{p_1, \ldots p_{2m}\}$ and replace the target and path conditions of these $m$ subformulas with these atoms. This produces a new formula $\gamma'$ that is amenable to lemma \ref{lemma:separation-linear-step3}. Applying the lemma produces a separated formula $\gamma''$ that uses the atoms $\{p_1, \ldots p_{2m}\}$. These atoms may appear under an $S$ in the separated formula $\gamma''$. Now, replace each of these atoms by the target/path condition they substituted earlier. This produces $\gamma'''$, a formula with the maximum $U$-nesting depth under an $S$ strictly $< n$. Applying the induction hypothesis on $\gamma'''$ proves this lemma.
    \end{description}
    \begin{remark*}
        We don't need to consider the value $m$ in our induction hypothesis, as required in the proof of lemma \ref{lemma:separation-linear-step2}.
    \end{remark*}
\end{proof}
Before we finally prove the separation theorem, notice that, since $U$ and $S$ are duals of each other, the eliminations in Section \ref{sec:eliminations-linear} and Lemmas \ref{lemma:separation-linear-step1}, \ref{lemma:separation-linear-step2}, \ref{lemma:separation-linear-step3} and \ref{lemma:separation-linear-step4} hold when the $U$ and $S$ are swapped.
%TODO: Understand and implement Theorem-Restate here
\begin{theorem}[Separation Property for Linear Time]
    \label{theorem:separation-linear-final}
    Any formula $\gamma$ that uses $S$ and $U$ can be separated.
\end{theorem}
\begin{proof}
    We induct over the \textit{junction depth} of the input formula. Define this depth as follows.
    \begin{definition}[Junction Depth]
        The junction depth of a temporal formula $\gamma$ is the length of the longest sequence of subformulas $\alpha_1, \alpha_2, \ldots \alpha_n$ of $\gamma$ such that
        \begin{enumerate}
            \item The root of all $\alpha_i$ is either a $U$ or an $S$.
            \item $\alpha_{i+1}$ is a subformula of $\alpha_i$.
            \item If $\alpha_i$ is rooted by a $U$ (or an $S$), then $\alpha_{i+1}$ is rooted by an $S$ (or a $U$, respectively).
            \item There is no subformula $\beta$ of $\gamma$ such that \label{item:junction-depth-extra-condition}
                \begin{enumerate}
                    \item $\beta$ is a strict subformula of $\alpha_i$.
                    \item $\alpha_{i+1}$ is a strict subformula of $\beta$.
                    \item $\beta$ and $\alpha_{i+1}$ are rooted by the same connective.
                \end{enumerate}
        \end{enumerate}
    \end{definition}
    Note that condition \ref{item:junction-depth-extra-condition} isn't necessary to compute the junction depth. However, I will use it in my proof.

    As an illustration, observe that the junction depth of the formula $U \left(a, S\left( U(c, d), U(e, f) \right) \right)$ is 3, and there are two possible sequences:
    \begin{itemize}
        \item $U \left(a, S\left( U(c, d), U(e, f) \right) \right), S(U(c, d), U(e, f)), U(c, d)$.
        \item $U \left(a, S\left( U(c, d), U(e, f) \right) \right), S(U(c, d), U(e, f)), U(e, f)$.
    \end{itemize}

    Let the junction depth of $\gamma$ be $n$.
    \begin{description}
        \item[Base case 1: $n = 1$.] The formula is already separated.
        \item[Base case 2: $n = 2$.] In this case, apply lemma \ref{lemma:separation-linear-step4} to separate $\gamma$.
        \item[Induction step: $n \geq 3$.] Let there be $m$ sequences of subformulas that witness the junction depth $n$. Form a set $A$ of all subformulas at position 3 of these $m$ sequences; the size of $A$ can be less than $m$. Note that condition \ref{item:junction-depth-extra-condition} makes these subformulas maximal; i.e., no formula in $A$ is a subformula of another. This maximality allows us to substitute each formula in $A$ with a newly introduced atom from the set $\{p_1, \ldots p_{|A|}\}$.

        Call the resulting formula $\gamma'$. Observe that this formula has a junction depth of strictly $< n$, allowing us to apply the induction hypothesis. This produces a separated formula $\gamma''$ with $|A|$ new atoms. Substitute the subformulas in $A$ at the corresponding atoms in $\gamma''$ to produce $\gamma'''$.

        Now, all subformulas in $A$ have a junction depth of $n - 2$. If all of these appear in the pure-present segment of $\gamma''$, the new junction depth of $\gamma'''$ grows to at most $n - 2$. Similarly, if one of these substitutions occurs inside a pure-past / pure-future segment of $\gamma''$, the junction depth grows to at most $n - 1$. This allows us to apply the induction hypothesis again, producing the fully separated formula $\gamma'''$.
    \end{description}
\end{proof}

\subsection{Implying Expressive-Completeness}

In this section, we provide an overview of the proof that \cref{theorem:separation-linear-final} implies expressive completeness. We start by proving an auxiliary result.
\begin{lemma}
    \label{lemma:pulling-out-present-lemma}
    Every formula with $m + 1$ free variables $\varphi(t, x_1, \ldots, x_m)$ in the first-order monadic logic of order with the monadic relations $\{Q_1, Q_2, \cdots Q_k\}$ can be written in the form
    \begin{equation*}
        \bigvee_{i=1}^l \beta_i (t) \land \alpha_i(t, x_1, \ldots, x_m)
    \end{equation*}
    for some $l$ where $t, x_1, \ldots, x_m$ are the $m + 1$ free variables in $\varphi$, $\beta_i(t)$ is quantifier free, and for all $i \in \{1, \ldots, l\}$ and $Q \in \{Q_1, \ldots, Q_k\}$, no atomic formula of the form $Q(t)$ appears in $\alpha_i(t, x_1, \ldots, x_m)$.
\end{lemma}
\begin{proof}
    We show this lemma by inducting on the quantifier depth of $\varphi$.
    \begin{description}
        \item[Base case.]
            $\varphi$ is quantifier free. This allows us to write $\varphi$ in DNF form. Each literal in the DNF form will be $\pm \alpha$, where $\alpha$ is an atomic formula in the vocabulary. We can specify all possible $\alpha$ using the following grammar notation.
            \begin{equation*}
                \begin{aligned}
                    \alpha &\Coloneqq Q(X) \mid X < X\\
                    Q &\Coloneqq Q_1 \mid \cdots \mid Q_k\\
                    X &\Coloneqq t \mid x_1 \mid \cdots \mid x_m
                \end{aligned}
            \end{equation*}
            After replacing cases like $t < t$ to $\bot$  and $t = t$ to $\top$, we group all $Q(t)$ to $\beta(t)$ and the rest to $\alpha(t, x_1, \ldots, x_m)$ in each disjunct in the DNF. This completes this case.

        \item[Induction case.] Suppose the lemma holds for all formulas of quantifier depth less than $n$, and suppose $\varphi$ has quantifier depth $n$. Begin by writing all subformulas of the form $\forall z.\, \alpha$ as $\lnot \exists z.\, \lnot \alpha$. After this transformation, $\varphi$ is a boolean combination of atomic formulas of the form $Q(y)$, $x < y$ for $x$ and $y$ in $\{t, x_1, \ldots, x_m\}$ and quantified subformulas $\exists z.\, \psi$ for some bound variable $z$.

        Observe that each $\psi$ in the previous form has a quantifier depth of $n-1$. Applying the induction hypothesis on $\psi$ gives us an equivalent formula
        \begin{equation*}
            \psi \equiv \bigvee_i \beta_i(t) \land \alpha_i(t, x_1, \ldots, x_m, z)
        \end{equation*}
        Now, we simply push the existential quantifier deeper inside $\exists z.\, \psi$:
        \begin{equation*}
            \begin{aligned}
                \exists z.\, \psi &\longmapsto \exists z.\, \left( \bigvee_i \beta_i(t) \land \alpha_i(t, x_1, \ldots, x_m, z) \right)\\
                &\longmapsto \bigvee_i \exists z.\, \left( \beta_i(t) \land \alpha_i(t, x_1, \ldots, x_m, z) \right)\\
                &\longmapsto \bigvee_i \beta_i(t) \land \exists z.\, \alpha_i(t, x_1, \ldots, x_m, z)\\
            \end{aligned}
        \end{equation*}
        where all $\beta_i(t)$ remain quantifier free and no $Q(t)$ appears in any $\alpha_i$.

        After writing each $\exists z.\, \psi$ in this form, $\varphi$ becomes a boolean combination of $Q(y)$, $x < y$ for all $x$ and $y$ in $\{t, x_1, \ldots, x_m\}$, and $\exists z.\, \alpha(t, x_1, \ldots x_m, z)$. We can now take the DNF form of this formula by treating each $\exists z.\, \alpha$ as though it were an atom. It isn't difficult to see that this final formula is what we need, proving the lemma.
    \end{description}
\end{proof}
Notice that the primary arguments in the proof of \cref{lemma:pulling-out-present-lemma} are quite general. These arguments can be reused to show similar results in the case of more complex first-order relational vocabularies. For now, consider a useful corollary.
\begin{corollary}
    \label{corollary:separate-present-form-fomlo}
    Every single-variable formula $\varphi(t)$ in the first-order monadic logic of order can be written in the form
    \begin{equation*}
        \bigvee_i \left( \beta_i(t) \land \bigwedge_j \left( \pm \exists y.\, \alpha_{i, j}\left(t, y\right) \right) \right)
    \end{equation*}
    where $\beta_i(t)$ is quantifier-free and $Q(t)$ doesn't appear in $\alpha$.
\end{corollary}
\begin{proof}
    This can easily be observed by realizing that, in the case of a single-variable formula, all $\alpha_i$ in \cref{lemma:pulling-out-present-lemma} must be boolean combination of formulas of the form $\exists y.\, \psi$. Considering each of these as atoms and writing the DNF form of the resulting formula gives us what we need.
\end{proof}
Finally, we consider the separation theorem.
\begin{theorem}[Separation Theorem for Linear Time]
    \label{theorem:separation-theorem-linear-time}
    Every single-variable formula $\varphi(t)$ of the first-order monadic logic of order with the monadic relations $\{Q_1, \ldots, Q_m\}$ evaluated over linear time can be expressed by a formula in the temporal logic of the strict $S$ and $U$ over linear flows of time.
\end{theorem}
\begin{proof}
    Before we begin the proof, note that, as per the established norms, the temporal language has access to the monadic relations $\{Q_1, \ldots, Q_m\}$ through the propositional atoms $\{q_1, \ldots, q_m\}$.

    We induct on the quantifier depth of $\varphi$.
    \begin{description}
        \item[Base case.] $\varphi(t)$ is quantifier free. Construct a temporal formula by replacing all instances of $Q_i(t)$ in $\varphi$ with the propositional atom $q_i$. This resulting formula is clearly equivalent to $\varphi$ when evaluated at any time point $t$. Notably, it's a \textit{pure-present} formula.
        \item[Induction case.] Suppose $\varphi(t)$ has quantifier depth $n$. Write $\varphi(t)$ in the form presented in \cref{corollary:separate-present-form-fomlo}.
        \begin{equation}
            \label{eq:present-separate-form-fomlo}
            \varphi(t) \equiv \bigvee_i \left( \beta_i(t) \land \bigwedge_j \left( \pm \exists y.\, \alpha_{i, j}\left(t, y\right) \right) \right)
        \end{equation}
        Observe that one can easily construct a pure-present temporal formula $\rho_i$ for each $\beta_i(t)$. Hence, we focus our attention on the $\exists y.\, \alpha(t, y)$. We start by getting rid of all instances of the variable $t$ in $\alpha$ by introducing a few new monadic relations.

        Introduce three new monadic symbols $R_<$, $R_=$, and $R_>$. In each $\alpha$, substitute all atomic formulas that involve $t$ in the following way.
        \begin{equation*}
            \begin{aligned}
                x < t &\longmapsto R_<(x)\\
                x = t &\longmapsto R_=(x)\\
                t < x  &\longmapsto R_>(x)
            \end{aligned}
        \end{equation*}
        By \cref{corollary:separate-present-form-fomlo}, these are the only instances of $t$ in $\alpha$. Call the resulting formula $\alpha'$. Transforming each $\alpha$ in $\varphi$ in this way produces the formula $\varphi'$, defined below
        \begin{equation}
            \label{eq:present-separate-form-fomlo-without-t}
            \varphi'(t) \triangleq \bigvee_i \left( \beta_i(t) \land \bigwedge_j \left( \pm \exists y.\, \alpha'_{i, j}\left(y\right) \right) \right)
        \end{equation}
        Observe that each $\alpha'$ in $\varphi'$ satisfies the following properties.
        \begin{alphaenumerate}
            \item Their quantifier depth is at most $n - 1$.
            \item They have a single free-variable ($y$).
            \item They are equivalent to $\alpha$ \textit{if} $R_<$, $R_=$, and $R_>$ are modelled \textit{appropriately} (we'll see what this means soon).
        \end{alphaenumerate}
        These conditions allow us to use the induction hypothesis on $\alpha'$ to produce a temporal formula $\gamma$. Notably, the increased pool of monadic relations has created the new propositional atoms $r_>$, $r_=$, and $r_<$. $\gamma$ may contain these atoms.

        Now, observe that the existential quantifier in $\exists y.\, \alpha'(y)$ can be expressed in the temporal language as $\diamond \gamma$, where $\diamond$ is shorthand for \textit{at some point in time}. $\diamond$ can be expressed with $S$ and $U$ as follows:
        \begin{equation*}
            \diamond \gamma \equiv \gamma \lor U(\gamma,\top) \lor S(\gamma, \top)
        \end{equation*}

        We proceed to construct temporal formulas $\gamma_{i, j}$ for each $\alpha'_{i, j}$ in \cref{eq:present-separate-form-fomlo-without-t}. This allows us to construct the monolithic temporal formula $\psi$:
        \begin{equation*}
            \psi \triangleq \bigvee_i \left( \rho_i \land \bigwedge_j \left( \pm \diamond \gamma_{i, j} \right) \right)
        \end{equation*}
        Again, if $r_<$, $r_=$, and $r_>$ are appropriately modelled, $\psi$ is equivalent to $\varphi(t)$.

        Using \cref{theorem:separation-linear-final}, we now \textit{separate} $\psi$ into a boolean combination of pure-past, present, and future formulas. We write the separated formula as follows:
        \begin{equation*}
            \psi \equiv \mathbb{B}(\psi_{<, 1}, \ldots, \psi_{<. m_<}, \psi_{=, 1}, \ldots, \psi_{<, m_=}, \psi_{>, 1}, \ldots \psi_{>, m_>})
        \end{equation*}
        where $\mathbb{B}$ abstracts the boolean combinations and the $\psi_{<, i}$, $\psi_{=, i}$, and $\psi_{>, i}$ are pure-past, present, and future formulas.

        We earlier stated that if $R_<$, $R_=$, and $R_>$ are appropriately modelled, $\psi$ is equivalent to $\varphi(t)$. The correct values of $R_<$, $R_=$, and $R_>$ are, quite naturally,
        \begin{equation*}
            \begin{aligned}
                R_< &= \{s \mid s < t\}\\
                R_= &= \{t\}\\
                R_> &= \{s \mid t < s\}
            \end{aligned}
        \end{equation*}
        Let $h$ be such an \textit{appropriate} assignment of atoms over the flow of time.

        Now, consider an assignment $h_<$ that agrees with $h$ on all atoms but $r_<$, $r_=$, and $r_>$. $h_<$ models $r_<$ to $\top$ everywhere, and $r_>$ and $r_=$ to $\bot$ everywhere. This assignment, by its definition, agrees with $h$ on the past of $t$, and consequently, for each pure-past $\psi_{<, i}$,
        \begin{equation*}
            h \vDash \psi_{<, i} \Longleftrightarrow h_< \vDash \psi_{<, i}
        \end{equation*}
        Construct the formula $\psi'_{<, i}$ by substituting all instances of $r_<$ in $\psi_{<, i}$ by $\top$ and all instances of $r_=$ and $r_>$ by $\bot$, i.e.,
        \begin{equation*}
            \psi'_{<, i} \triangleq \psi_{<, i} \left[ \begin{smallmatrix}
                r_< \mapsto \top\\
                r_= \mapsto \bot\\
                r_> \mapsto \bot
            \end{smallmatrix} \right]
        \end{equation*}
        It's easy to see that
        \begin{equation*}
            h_< \vDash \psi_{<, i} \Longleftrightarrow h_< \vDash \psi'_{<, i}
        \end{equation*}
        Hence,
        \begin{equation*}
            h \vDash \psi_{<, i} \Longleftrightarrow h_< \vDash \psi_{<, i} \Longleftrightarrow h_< \vDash \psi'_{<, i}
        \end{equation*}
        Observe that $\psi'_{<, i}$ no longer uses the additional atoms! And since $h$ and $h_<$ agree on all other atoms,
        \begin{equation*}
            h \vDash \psi_{<, i} \Longleftrightarrow h \vDash \psi'_{<, i}
        \end{equation*}
        We can similarly substitute the atoms in the $\psi_{>, i}$ and $\psi_{=, i}$ to get rid of $r_<$, $r_=$, and $r_>$ in $\psi$. Call this new formula $\psi'$.
        \begin{equation*}
            \psi' \triangleq \mathbb{B}(\psi'_{<, 1}, \ldots, \psi'_{<. m_<}, \psi'_{=, 1}, \ldots, \psi'_{<, m_=}, \psi'_{>, 1}, \ldots \psi'_{>, m_>})
        \end{equation*}
        We claim that $\psi'$ is equivalent to $\varphi(t)$. To see why, take any linear temporal structure $\mathcal{M} = (T, <, h)$ and a point $t \in T$ in the flow. Let $h'$ be the extension of $h$ with the appropriate valuations of $R_<$, $R_=$ and $R_>$ and $\mathcal{M}'$ be $(T, <, h')$. It's easy to see that the following double-implications immediately hold:
        \begin{equation*}
            \begin{aligned}
                \mathcal{M}, t \vDash \varphi(t) \;\Longleftrightarrow\; \mathcal{M}', t \vDash \varphi'(t)
                \;\Longleftrightarrow\; \mathcal{M}', t \vDash \psi
                \;\Longleftrightarrow\; \mathcal{M}, t \vDash \psi'
            \end{aligned}
        \end{equation*}
        This proves the separation theorem.
    \end{description}
\end{proof}

\section{Ordered Trees}
\label{sec:ordered-trees}

% TODO: Think up a better introduction of ordered trees.

Flows of time can be more complicated than the linear structures we've seen so far. The notion of branching time, where the flow resembles a tree, is a well known example. While temporal languages over unordered trees have been studied quite extensively (see \cite{RabinovichUnordered00}), in this work we look at ordered trees.

In addition to the descendent order (which corresponds to the natural forward flow of time), ordered trees use a \textit{sibling} order. Correspondingly, the first-order vocabulary includes two binary relations: $<$ and $\prec$, where $x < y$ indicates $y$ is a descendant of $x$ and $x \prec y$ indicates $y$ comes after $x$ in the sibling order. All immediate children of a node are totally ordered by $\prec$. %TODO: Consider adding an axiom that describes the limits of $\prec$ and $<$.

In \cite{xpathComplete}, Marx introduced the temporal language $\mathcal{X}_{until}$ over ordered trees. This language has the same expressive power as \textit{Conditional XPath}, which Marx proved to be expressively complete in \cite{marx2005conditional}. It defines four connectives that are similar to the strict $U$ and $S$ Gabbay defines for linear time. These are $\Larrow$, $\Rarrow$, $\Uarrow$, and $\Darrow$, defined by the following monadic first-order formulas.
\begin{equation*}
    \begin{aligned}
        \varphi_{\Darrow}(t, X_1, X_2) &\triangleq \exists x.\, \left[ \left( t < x \right) \land X_1(x) \land \forall y \left( \left( t < y < x \right) \to X_2(y) \right) \right]\\
        \varphi_{\Uarrow}(t, X_1, X_2) &\triangleq \exists x.\, \left[ \left( x < t \right) \land X_1(x) \land \forall y \left( \left( x < y < t \right) \to X_2(y) \right) \right]\\
        \varphi_{\Rarrow}(t, X_1, X_2) &\triangleq \exists x.\, \left[ \left( t \prec x \right) \land X_1(x) \land \forall y \left( \left( t \prec y \prec x \right) \to X_2(y) \right) \right]\\
        \varphi_{\Larrow}(t, X_1, X_2) &\triangleq \exists x.\, \left[ \left( x \prec t \right) \land X_1(x) \land \forall y \left( \left( x \prec y \prec t \right) \to X_2(y) \right) \right]
    \end{aligned}
\end{equation*}
Marx suggested a separation property for this temporal language over ordered trees. The regions he proposed, with respect to an arbitrary point $t$ in the flow, were
\begin{itemize} %TODO: Describe Marx's regions here.
    \item The \textit{present} point, which we call $t$.
    \item The \textit{future}, defined as $\{x \mid t < x \}$.
    \item The \textit{left} of $t$, defined as $\{x \mid x \prec t \lor \exists y.\, y \prec t \land y < x \}$.
    \item The \textit{right} of $t$, defined analogously as $\{x \mid t \prec x \lor \exists y.\, t \prec y \land y < x \}$.
    \item The \textit{past}, which consists of all points not claimed by other regions.
\end{itemize}
\cref{fig:marx-regions} shows how these regions partition the tree.
\begin{figure}[h]
    \centering
    \tikzfig{separation-tree-marx}
    \caption[]{\emph{Marx's regions.} The black node is the present, the \textcolor{orange}{orange} nodes belong to the \textit{left} region, the \textcolor{red}{red} nodes to the \textit{right}, the \textcolor{blue}{blue} nodes are the \textit{future}, and the \textcolor{OliveGreen}{green} nodes are the \textit{past}. The descendant relation is given by the black lines and the sibling order is denoted by the \textcolor{red}{red} lines. Dashed lines indicate potential intermediate nodes.}
    \label{fig:marx-regions}
\end{figure}

Unfortunately, Marx's proof of separation in \cite{xpathComplete} is incorrect. He fails to take into consideration that the $\Darrow$ modality is non-deterministic. This indicates that one cannot extend \cref{eq:or-and-S-U} to $\Darrow$, as
\begin{equation*}
    \Darrow(a, b \land c) \not\equiv \Darrow(a, b) \land \Darrow(a, c)
\end{equation*}
This is made explicit in \cref{fig:darrow-a-b-and-c}.
\begin{figure}[h]
    \centering
    \tikzfig{darrow-a-b-and-c}
    \caption[]{This is an example that satisfies $\Darrow(a, b)$ and $\Darrow(a, c)$, but not $\Darrow(a, b \land c)$.}
    \label{fig:darrow-a-b-and-c}
\end{figure}

We present our results in the next few sections. In \cref{sec:partial-separation-start}, we describe how a class of $\mathcal{X}_{until}$ formulas can be separated. We believe this class is the largest class of formulas that can be separated. In \cref{sec:ef-games-start}, we present the arguments behind our belief that certain formulas (outside the previously mentioned class) can never be separated. Separately, in \cref{sec:finer-partition}, we show that a more desirable partitioning of the tree only leads back to Marx's regions, justifying them somewhat.

\subsection{Partial Separation of $\mathcal{X}_{until}$}
\label{sec:partial-separation-start}

As with linear flows, we can define a notion of syntactically pure formulas in $\mathcal{X}_{until}$. Again, to simplify presentation, we refer to formulas rooted by a $\pi$ for $\pi \in \{\Larrow, \Rarrow, \Uarrow, \Darrow\}$ as a $\pi$-formula.
\begin{definition}
    A temporal formula $\varphi$ in $\mathcal{X}_{until}$ is
    \begin{enumerate}
        \item syntactically pure-present if it doesn't use any connectives from $\{\Larrow, \Rarrow, \Uarrow, \Darrow\}$.
        \item syntactically pure-future if it's a boolean combination of $\Darrow$-formulas that don't use the $\Uarrow$ connective.
        \item syntactically pure-left if it's a boolean combination of $\Larrow$-formulas that contain pure-present, pure-future and/or smaller $\Larrow$-formulas of this form.
        \item syntactically pure-right if it's a boolean combination of $\Rarrow$-formulas that contain pure-present, pure-future, and/or smaller $\Rarrow$-formulas of this form.
        \item syntactically pure-past if it's a boolean combination of $\Uarrow$-formulas that contain pure-present, pure-left, pure-right, and/or smaller $\Uarrow$-formulas of this form.
    \end{enumerate}
\end{definition}

It's simple enough to observe that all syntactically pure formulas are semantically pure. To simplify our arguments, we also introduce the notion of the \textit{pure $\pi$ formula} for $\pi \in \{ \Larrow, \Rarrow, \Darrow, \Uarrow \}$. Simply put, a formula is pure $\pi$ \textit{iff} the only connective it uses is $\pi$.

In this subsection, we aim to show the following claim.
\begin{claim*}[Partial Separation of $\mathcal{X}_{until}$]
    \label{claim:partial-separation-trees}
    Let $\varphi$ be a formula in $\mathcal{X}_{until}$ such that all $\Darrow$-subformulas of $\varphi$ are syntactically pure-future. Then, $\varphi$ can be separated.
\end{claim*}
This claim can be seen as a consequence of two facts. First, \cref{eq:or-and-S-U} is valid for the modalities $\pi \in \{\Uarrow, \Larrow, \Rarrow\}$. Second, the eliminations presented in \cite{xpathComplete} prove that formulas of the form $\Uarrow(a \land \pm \Darrow(p, q), b \lor \pm \Darrow(p, q))$ can be separated.

\subsubsection{Separating $\Larrow$, $\Rarrow$ and $\Uarrow$}

We start with a few lemmas.
\begin{lemma}
    \label{lemma:partial-separation-trees-step1}
    Let $\varphi$ be a formula in $\mathcal{X}_{until}$ that doesn't use the $\Uarrow$ and $\Darrow$ connectives. Then, $\varphi$ can be separated.
\end{lemma}
\begin{proof}
    This is a simple consequence of \cref{theorem:separation-linear-final}. Observe that $\varphi$ can only use the $\Larrow$ and $\Rarrow$ connectives, and that the directions of these connectives prevent $\varphi$ from probing nodes that aren't siblings of the present point $t$. Moreover, their operation mirrors that of $S$ and $U$ over linear time. Since the sibling order $\prec$ over all siblings of a node produces a total-order, we're justified in applying \cref{theorem:separation-linear-final}.
\end{proof}
Next, we lift a few eliminations from \cite{xpathComplete}.
\begin{lemma}
    \label{lemma:partial-separation-trees-step2}
    The following eliminations are valid.
    \begin{itemize}
        \item $\Larrow(\alpha \land \mathord{\Uarrow}(\varphi, \psi), \beta) \equiv \Larrow(\alpha, \beta) \land \Uarrow(\varphi, \psi)$
        \item $\Larrow(\alpha \land \lnot \Uarrow (\varphi, \psi), \beta) \equiv \Larrow(\alpha, \beta) \land \lnot \Uarrow(\varphi, \psi)$
        \item $\Larrow(\alpha, \beta \lor \Uarrow(\varphi, \psi)) \equiv \Larrow(\alpha, \beta) \lor \left( \Larrow(\alpha, \top) \land \Uarrow(\varphi, \psi) \right)$
        \item $\Larrow(\alpha, \beta \lor \lnot \Uarrow(\varphi, \psi)) \equiv \Larrow(\alpha, \beta) \lor \left( \Larrow(\alpha, \top) \land \lnot \Uarrow(\varphi, \psi) \right)$
    \end{itemize}
\end{lemma}
\begin{proof}
    These equivalences follow from the fact that the truth of $\pm \Uarrow(\varphi, \psi)$ at a sibling of $t$ implies $\pm \Uarrow(\varphi,\psi)$ at $t$. The eliminations merely explicate this fact.
\end{proof}
\begin{note*}
    These eliminations \textit{don't} proliferate the parameters of the $\Uarrow$ connective. The eliminations we observed in \cref{sec:eliminations-linear} don't have this property.
\end{note*}
We now implement these eliminations in a more general setting.
\begin{lemma}
    \label{lemma:partial-separation-trees-step3}
    Let $\alpha$ and $\beta$ be boolean combinations of propositional atoms from $\mathcal{P}$ and instances of $\Uarrow(\varphi, \psi)$, for \emph{any} $\mathcal{X}_{until}$ formulas $\varphi$ and $\psi$. Then, $\Larrow(\alpha, \beta)$ can be written as a boolean combination of pure-$\Larrow$ formulas and $\Uarrow(\varphi, \psi)$.
    \begin{remark*}
        This lemma only yields a separated formula if $\Uarrow(\varphi, \psi)$ is a pure-past formula.
    \end{remark*}
\end{lemma}
\begin{proof}
    Since $\Larrow$ and $\Rarrow$ satisfy a version of \cref{eq:or-and-S-U}, we can follow the procedure outlined in \cref{lemma:separation-linear-step1} and write $\Larrow(\alpha, \beta)$ in the manner of \cref{eq:dnf-form-S-alpha-beta} to get
    \begin{equation*}
        \Larrow(\alpha, \beta) \equiv \bigvee_i \bigwedge_j \Larrow\left(\alpha_{i, 1} \land \cdots \land \alpha_{i, m_i}, \beta_{j, 1} \lor \cdots \lor \beta_{j, n_i} \right)
    \end{equation*}
    As in \cref{lemma:separation-linear-step1}, the $\Larrow$-formula for each $\{i, j\}$ can be considered to be in the form of
    \begin{equation*}
        \begin{aligned}
            &\Larrow(\alpha' \land \pm \Uarrow(\varphi, \psi), \beta')\\
            &\Larrow(\alpha', \beta' \lor \pm \Uarrow(\varphi, \psi))\\
            &\Larrow(\alpha' \land \pm \Uarrow(\varphi, \psi), \beta' \lor \pm \Uarrow(\varphi, \psi))\\
        \end{aligned}
    \end{equation*}
    We can apply the eliminations detailed in \cref{lemma:partial-separation-trees-step2} to complete the proof.
\end{proof}
We now generalize this lemma a little further.
\begin{lemma}
    \label{lemma:partial-separation-trees-step4}
    Take an expanded set of atoms $\mathcal{P}' \triangleq \mathcal{P} \cup \{\Uarrow(\varphi, \psi)\}$. Suppose $\gamma$ is a formula in $\mathcal{X}_{until}$ that only uses the $\Larrow$ connective and takes atoms from $\mathcal{P}'$. Then, $\gamma$ is equivalent to a boolean combination of pure-$\Larrow$ formulas, atoms in $\mathcal{P}$, and the formula $\Uarrow(\varphi, \psi)$.
\end{lemma}
\begin{proof}
    Despite the stricter wording, this lemma is the counterpart of \cref{lemma:separation-linear-step2} with the connectives $S$ and $U$ substituted by $\Larrow$ and $\Uarrow$. As with that lemma, we induct on $(n_1, n_2)$, where $n_1$ is the highest $\Larrow$-depth at which a $\Uarrow(\varphi, \psi)$ appears in $\gamma$ and $n_2$ is the number of instances of $\Uarrow(\varphi, \psi)$ at depth $n_1$.
    \begin{description}
        \item[Base case.] $(n_1, n_2) = (1, 1)$. This is equivalent to \cref{lemma:partial-separation-trees-step3}.
        \item[Induction step.] Take a subformula $\Larrow(\alpha', \beta')$ at $\Larrow$-depth $(n-1)$ such that (1) $\Larrow$ doesn't appear in $\alpha'$ and $\beta'$, and (2) $\Uarrow(\varphi, \psi)$ appears in at least one of $\alpha'$ and $\beta'$. It's easy to see that applying \cref{lemma:partial-separation-trees-step3} to $\Larrow(\alpha', \beta')$ strictly reduces $(n_1, n_2)$, allowing us to apply the induction hypothesis.
    \end{description}
    \begin{remark*}
        Since $\varphi$ and $\psi$ aren't proliferated in \cref{lemma:partial-separation-trees-step2}, they don't enter the pure-$\Larrow$ formulas in this lemma.
    \end{remark*}
\end{proof}
We now consider the case of multiple $\Uarrow$-formulas inside a $\Larrow$-formula.
\begin{lemma}
    \label{lemma:partial-separation-trees-step5}
    Take an expanded set of atoms $\mathcal{P}' \triangleq \mathcal{P} \cup \{\Uarrow(\varphi_1, \psi_1), \ldots, \Uarrow(\varphi_n, \psi_n)\}$. Suppose $\gamma$ is a formula in $\mathcal{X}_{until}$ that only uses the $\Larrow$ connective and takes atoms from $\mathcal{P}'$. Then, $\gamma$ is equivalent to a boolean combination of pure-$\Larrow$ formulas, atoms in $\mathcal{P}$, and the $\Uarrow(\varphi_i, \psi_i)$ formulas.
\end{lemma}
\begin{proof}
    This is the counterpart of \cref{lemma:separation-linear-step3}. Predictably, we prove this by inducting on $n$. The details are left to the reader. %TODO: Ask if this is lazy in the meeting.
    \begin{remark*}
        Again, the structure of the $\Uarrow$-formulas is maintained during the transformation.
    \end{remark*}
\end{proof}
Before we proceed with the next result, note that Lemmas \ref{lemma:partial-separation-trees-step3}, \ref{lemma:partial-separation-trees-step4}, and \ref{lemma:partial-separation-trees-step5} are valid when the $\Larrow$ is replaced by a $\Rarrow$.
\begin{lemma}
    \label{lemma:partial-separation-trees-step6}
    Take an expanded set of atoms $\mathcal{P}' \triangleq \mathcal{P} \cup \{\Uarrow(\varphi_1, \psi_1), \ldots, \Uarrow(\varphi_n, \psi_n)\}$. Suppose $\gamma$ is a formula in $\mathcal{X}_{until}$ that only uses the $\Larrow$ and $\Rarrow$ connectives and takes atoms from $\mathcal{P}'$. Then, $\gamma$ is equivalent to a boolean combination of pure-$\Larrow$ formulas, pure-$\Rarrow$ formulas, atoms in $\mathcal{P}$, and the $\Uarrow(\varphi_i, \psi_i)$.
\end{lemma}
\begin{proof}
    Introduce new atoms $r_1, \ldots, r_n$ to $\mathcal{P}$ to produce $\mathcal{P}''$. In $\gamma$, replace each instance of $\Uarrow(\varphi_i, \psi_i)$ with $r_i$ to produce the formula $\gamma'$. Notice that $\gamma'$ has no $\Uarrow$-subformulas. Separate $\gamma'$ according to \cref{lemma:partial-separation-trees-step1} to get
    \begin{equation*}
        \gamma' \equiv \mathbb{B}(\gamma_{\Larrow, 1}, \ldots, \gamma_{\Larrow, n_{\Larrow}}, \gamma_{\Rarrow, 1}, \ldots, \gamma_{\Rarrow, n_{\Rarrow}}, \gamma_{=, 1}, \ldots \gamma_{=, n_=})
    \end{equation*}
    where all $\gamma_{\Larrow, i}$ formulas only use the $\Larrow$ connective, all $\gamma_{\Rarrow, j}$ formulas only use the $\Rarrow$ connective, and all $\gamma_{=, k}$ use no connectives.. At this stage, replace all instances of $r_i$ by $\Uarrow(\varphi_i, \psi_i)$ in the $\gamma_{\Larrow, i}$, $\gamma_{\Rarrow, j}$, and $\gamma_{=, k}$ to produce $\gamma'_{\Larrow, i}$, $\gamma'_{\Rarrow, j}$ and $\gamma'_{=, k}$. The $\gamma'_{=, k}$ already satisfy our condition; hence, we only apply \cref{lemma:partial-separation-trees-step5} to the others to prove this lemma.
\end{proof}
We can now state an interesting corollary.
\begin{corollary}
    \label{corollary:partial-separation-without-down}
    Let $\gamma$ be a formula in $\mathcal{X}_{until}$ that doesn't use the $\Darrow$ connective. Then, $\gamma$ can be separated.
\end{corollary}
\begin{proof}
    This is a simple matter of noticing that all $\Uarrow$-formulas that don't use the $\Darrow$ connective are syntactically pure-past and applying \cref{lemma:partial-separation-trees-step6}.
\end{proof}

\subsubsection{Separating $\Darrow$, $\Rarrow$, and $\Larrow$}
At this stage, we begin considering pure-future $\Darrow$-formulas. It's simple to notice that any formula that only uses the $\Larrow$ (or $\Rarrow$) and $\Darrow$ connectives is immediately syntactically separated; it is a boolean combination of syntactically pure-left (or pure-right) and pure-future formulas. It's easy to extend this observation to show the following lemma.
\begin{lemma}
    \label{lemma:partial-separation-trees-step7}
    Let $\gamma$ be a formula in $\mathcal{X}_{until}$ that doesn't use the $\Uarrow$ connective. Then, $\gamma$ can be separated.
\end{lemma}
\begin{proof}
    Let $\Darrow(\varphi_1, \psi_1), \ldots, \Darrow(\varphi_n, \psi_n)$ be $\gamma$'s $\Darrow$-subformulas that don't appear under ths scope of a $\Darrow$ (i.e., they're the \textit{top-level} $\Darrow$-subformulas). Introduce new atoms $\{r_1, \ldots, r_n\}$ and for each $i \in \{1, \ldots, n\}$, replace the instance of the subformula $\Darrow(\varphi_i, \psi_i)$ in $\gamma$ with $r_i$ to produce the formula $\gamma'$.

    Observe that $\gamma'$ only uses the $\Larrow$ and $\Rarrow$ connectives. This allows us to use \cref{lemma:partial-separation-trees-step1} separate $\gamma'$. In the separated formula, substitute all instances of $r_i$ with $\Downarrow_i(\varphi_i, \psi_i)$. It's easy to see that, after substitution, we get a syntactically separated formula.
\end{proof}
We can extend this reasoning further by reusing the method used to prove \cref{lemma:partial-separation-trees-step6}.
\begin{corollary}
    \label{lemma:partial-separation-trees-step8}
    Let $\gamma$ be a $\mathcal{X}_{until}$ formula such that all $\Darrow$ subformulas are syntactically pure-future and all $\Uarrow$-subformulas are syntactically pure-past. Then, $\gamma$ can be separated.
\end{corollary}
\begin{proof}
    Let $\Darrow(\varphi_1, \psi_1), \ldots, \Darrow(\varphi_n, \psi_n)$ be all subformulas of $\gamma$ that (1) don't appear under a $\Darrow$ and (2) don't appear under a $\Uarrow$. Note that, as all $\Uarrow$-subformulas are syntactically pure-past, any $\Darrow$-subformula that appears under a $\Uarrow$ must have an $\Larrow$ or $\Rarrow$ between it and the $\Uarrow$. Also, note that each $\Darrow(\varphi_i, \psi_i)$ are syntactically pure-future. %TODO: Possible diagram here.

    Introduce $n$ new atoms $r_1, \ldots r_n$ and substitute each $\Darrow(\varphi_i, \psi_i)$ in $\gamma$ by $r_i$. Note that no $r_i$ is embedded under a $\Uarrow$. Call this new formula $\gamma'$. It's easy to observe that one can apply \cref{lemma:partial-separation-trees-step6} to separate $\gamma'$. Let the separated formula be $\gamma''$. Since no $r_i$ is under a $\Uarrow$ in $\gamma'$, no $r_i$ is under a $\Uarrow$ in $\gamma''$.

    Hence, all $r_i$ must occur as a pure-present atom or inside a pure-$\Larrow$ formula or a pure-$\Rarrow$ formula in $\gamma''$. Substituting $\Darrow(\varphi_i, \psi_i)$ for each $r_i$ keeps the formula separated, proving the lemma.
\end{proof}

%\subsubsection{Normal Form} % Consider deleting
%We've reached another checkpoint in our proof of \cref{claim:partial-separation-trees}. At this stage, it's clear that only the case where the pure-future $\Darrow$-formula is inside an $\Uarrow$-formula remains. To facilitate progress, we consider a \textit{standard form} for pure-past $\Uarrow$-formulas.
%\begin{definition}
%    Let $\gamma = \Uarrow(\varphi, \psi)$ be a formula in $\mathcal{X}_{until}$ such that $\varphi$ and $\psi$ are of the form
%    \begin{equation*}
%        \begin{aligned}
%            \varphi \triangleq \bigwedge_i \pm \varphi_i && \psi \triangleq \bigvee_j \pm \psi_j
%        \end{aligned}
%    \end{equation*}
%    where $\varphi_i$ and $\psi_j$ are either atoms, pure-left $\Larrow$-formulas, pure-right $\Rarrow$-formulas, or smaller $\Uarrow$-formulas in standard form. Then, $\gamma$ is a $\Uarrow$-formula in standard form.
%\end{definition}
%It's easy to see that all standard-form $\Uarrow$-formulas are syntactically pure-past. However, the next result makes these more useful.
%\begin{lemma}
%    \label{lemma:partial-separation-trees-step9}
%    Let $\gamma$ be a syntactically pure-past formula in $\mathcal{X}_{until}$. Then, $\gamma$ can be written as a boolean combination of $\Uarrow$-formulas in standard form.
%\end{lemma}
%\begin{proof}
%    Since $\gamma$ is a boolean combination of syntactically pure-past $\Uarrow$-formulas, we limit our discussion to the case where $\gamma$ is a $\Uarrow$ formula. Hence,
%    \begin{equation*}
%        \gamma \triangleq \Uarrow(\varphi, \psi)
%    \end{equation*}
%    We begin by inducting on the $\Uarrow$-depth of $\gamma$.
%    \begin{description}
%        \item[Base case.] The $\Uarrow$-depth of $\gamma$ is 1. This means $\varphi$ and $\psi$ only use the $\Larrow$, $\Rarrow$, and $\Darrow$ connectives. We hence use \cref{lemma:partial-separation-trees-step7} to separate them to $\varphi'$ and $\psi'$. Since $\gamma$ is a syntactically pure-past formula, $\varphi'$ and $\psi'$ will be boolean combinations of atoms, pure-left $\Larrow$-formulas, pure-right $\Rarrow$-formulas. Writing $\varphi'$ and $\psi'$ in DNF and CNF form gives us
%        \begin{equation*}
%            \begin{aligned}
%                \varphi' &\equiv \bigvee_i \bigwedge_j \pm \varphi'_{i, j}\\
%                \psi' &\equiv \bigwedge_i \bigvee_j \pm \psi'_{i, j}
%            \end{aligned}
%        \end{equation*}
%        where each $\varphi'_{i, j}$ and $\psi'_{i, j}$ are pure-left $\Larrow$-formulas, pure-right $\Rarrow$-formulas, or atoms. Hence, by token of the procedure used in \cref{eq:dnf-form-S-alpha-beta}, we have
%        \begin{equation*}
%            \begin{aligned}
%                \Uarrow(\varphi, \psi) \equiv \bigvee_i \bigwedge_j \Uarrow(\pm \varphi'_{i, 1} \land \cdots \land \pm \varphi'_{i, n_i} , \pm \psi'_{j, 1} \lor \cdots \lor \pm \psi'_{j, m_j})
%            \end{aligned}
%        \end{equation*}
%        This final form is clearly in standard form.
%
%        \item[Induction step.] Let $\Uarrow(\alpha, \beta), \ldots, \Uarrow(\alpha, \beta)$ be all of $\varphi$'s and $\psi$'s $\Uarrow$-subformulas that aren't under the scope of a $\Uarrow$. Since these subformulas have a strictly lower $\Uarrow$-depth than $\gamma$, we can apply the induction hypothesis to produce equivalent boolean combinations of $\Uarrow$-formulas in standard form for each $\Uarrow(\alpha_i, \beta_i)$. Simply substitute these formulas for $\Uarrow(\alpha_i, \beta_i)$ in $\varphi$ and $\psi$ to produce formulas $\varphi'$ and $\psi'$ in which each $\Uarrow$-subformula is in standard form. This gives us the new formula $\gamma'$ equivalent to $\gamma$:
%        \begin{equation*}
%            \gamma' \triangleq \Uarrow(\varphi', \psi')
%        \end{equation*}
%        Notice that the separation procedure described in the proof of \cref{lemma:partial-separation-trees-step8} maintains the structure of the pure-past $\Uarrow$-subformulas in the separated form. Directly applying that lemma on $\varphi'$ and $\psi'$ produces the separated formulas $\varphi''$ and $\psi''$. As in the base case, we can write $\varphi''$ in DNF and $\psi''$ in CNF to complete the proof.
%    \end{description}
%\end{proof}

\subsubsection{Pulling out $\Darrow$ from $\Uarrow$}
\label{sec:eliminations-trees}

We now restate some of the eliminations justified in the appendix of \cite{xpathComplete}. These eliminations make vital use of the formula $\theta$, defined as
\begin{equation}
    \label{eq:theta-defn}
    \theta \triangleq \Larrow\left(\varphi \lor \left( \psi \land \Darrow\left( \varphi, \psi \right) \right), \top \right) \lor \Rarrow \left( \varphi \lor \left( \psi \land \Darrow \left( \varphi, \psi \right) \right), \top \right)
\end{equation}
$\theta$ merely states that ``my parent satisfies $\Darrow(\varphi, \psi)$ because of a sibling of mine.'' Hence, $\varphi \lor \theta$ implies $\Darrow(\varphi, \psi)$ at the parent \textit{(note the one-way implication)}. Similarly, $\lnot \theta$ states that ``no sibling of mine is responsible for my parent satisfying $\Darrow(\varphi, \psi)$.'' Consequently, $\lnot \varphi \land \lnot \psi \land \lnot \theta$ implies $\lnot \Darrow(\varphi, \psi)$ at the parent. Notably, $\theta$ is a disjunction of a pure-left and a pure-right formula, and hence can appear inside the scope of a $\Uarrow$ in a pure-past formula.

Now, we move onto the eliminations.

\paragraph*{$\blacktriangleright\;\, \Uarrow(\alpha \land \Darrow(\varphi, \psi), \beta)$}

This formula requires $\Darrow(\varphi, \psi)$ to be true at the ancestor node that satisfies $\alpha$. The path taken by this $\Darrow$-formula can deviate from the ancestral path to $\alpha$ at any point. With $\theta$, we can measure when it deviates. Hence, this formula is equivalent to
\begin{equation*}
    \begin{aligned}
        &\Uarrow(\beta \land (\theta \lor \varphi) \land \Uarrow(\alpha, \beta \land \psi), \beta) \\
        \lor \quad &\Uarrow(\alpha, \beta \land \psi) \land \left( \theta \lor \varphi \lor \left( \psi \land \Darrow(\varphi, \psi) \right) \right) \\
    \end{aligned}
\end{equation*} \lipicsEnd

\paragraph*{$\blacktriangleright\;\, \Uarrow(\alpha \land \lnot \Darrow(\varphi, \psi), \beta)$}

We again have an ancestral path to $\alpha$, and at that point $\Darrow(\varphi, \psi)$ cannot be true. This indicates that $\Darrow(\varphi, \psi)$ cannot be true along this ancestral path as well, indicating that, we must either hit a point on the path where $\lnot \varphi \land \lnot \psi$ is true, or we need to force $\lnot \Darrow(\varphi, \psi)$ at the present. This can be observed in the following separated formula.
\begin{equation*}
    \begin{aligned}
        &\Uarrow(\lnot \theta \land \lnot \varphi \land \lnot \psi \land \beta \land \Uarrow(\alpha, \beta \land \lnot \varphi \land \lnot \theta), \beta)\\
        \lor \quad &\Uarrow(\alpha, \beta \land \lnot \varphi \land \lnot \theta) \land \lnot \theta \land ((\lnot \varphi \land \lnot \psi) \lor (\lnot \varphi \land \lnot \Darrow(\varphi, \psi)))
    \end{aligned}
\end{equation*} \lipicsEnd

\paragraph*{$\blacktriangleright\;\, \Uarrow(\alpha, \beta \lor \Darrow(\varphi, \psi))$}

As in \cref{sec:eliminations-linear}, we use the idea of the \textit{unfulfilled} point. This time, such points are detected by noticing that $\lnot \varphi \land \lnot \theta$ are true along the ancestral path to a $\lnot \beta$. We attempt to fulfil the point by ensuring $(\varphi \lor \theta) \lor \psi$. This gives us the entire formula
\begin{equation*}
    \begin{aligned}
        &\Uarrow(\alpha, \lnot \alpha \land (\Uarrow(\lnot \beta \land \lnot \alpha, \lnot \alpha \land \lnot \varphi \land \lnot \theta) \to (\psi \lor \varphi \lor \theta)))\\
        \land \quad &\Uarrow(\lnot \beta \land \lnot \alpha, \lnot \alpha \land \lnot \varphi \land \lnot \theta) \to (\varphi \lor \theta \lor (\psi \land \Darrow(\varphi, \psi)))
    \end{aligned}
\end{equation*} \lipicsEnd

\paragraph*{$\blacktriangleright\;\, \Uarrow(\alpha, \beta \lor \lnot \Darrow(\varphi, \psi))$}

Again, as in \cref{sec:eliminations-linear}, we use the idea of a dangerous point. We look for ancestral paths to a $\lnot \beta$ that satisfy $\psi$ at each point. If we find such a path, we enforce $\lnot \varphi \land \lnot \theta$. This gives the formula
\begin{equation*}
    \begin{aligned}
        &\Uarrow(\alpha, \lnot \alpha \land (\Uarrow(\lnot \alpha \land \lnot \beta, \psi \land \lnot \alpha) \to (\lnot \varphi \land \lnot \theta)))\\
        \land \quad &\Uarrow(\lnot \alpha \land \lnot \beta, \psi \land \lnot \alpha) \to (\lnot \theta \land \lnot \varphi \land (\lnot \psi \lor \lnot \Darrow(\varphi, \psi)))
    \end{aligned}
\end{equation*} \lipicsEnd

In the same vein, we can separate any combination of $\Uarrow(\alpha \pm \Darrow(\varphi, \psi), \beta \lor \pm \Darrow(\varphi, \psi))$. We refer to \cite{xpathComplete} for the full arguments. Notably, if $\alpha$, $\beta$, $\varphi$, and $\psi$ were replaced by atoms, the formulas on the right would be syntactically separated. And if $\theta$ was also replaced by an atom, the formulas on the right only use the $\Uarrow$ and $\Darrow$ connectives. Observe that the only instance of $\Darrow$ in all cases would be $\Darrow(\varphi, \psi)$.

Unfortunately, unlike the eliminations in \cref{lemma:partial-separation-trees-step2}, the parameters of the $\Darrow$ (the $\varphi$ and $\psi$) appear outside of the $\Darrow$ in the separated formula. This complicates our proof.

% Regardless, the only $\Darrow$ subformula in the separated variant is $\Darrow(\varphi, \psi)$. % TODO: NOT TRUE. IF $\varphi$ and $\psi$ appear under a $\Larrow$, then we might have to deal with a different $\Darrow$ formula.

\subsubsection{Final steps}

In the following discussion, we consider $\Uarrow$-formulas that are \textit{almost} pure-past. These are $\Uarrow$-formulas wherein the arguments of every $\Uarrow$-subformula \textit{(including the full formula)} consist of boolean combinations of pure-left $\Larrow$ formulas, pure-right $\Rarrow$ formulas, $\Uarrow$ formulas, and $\Darrow$ formulas. We will later show how to write every $\Uarrow$ formula in this way. In our proof, the $\Darrow$ formulas increase in complexity until we reach our target.

We begin with the following lemma.
\begin{lemma}
    \label{lemma:partial-separation-trees-step10}
    Let $\varphi$ and $\psi$ be boolean combinations of atoms, pure-$\Larrow$, and pure-$\Rarrow$ formulas. Let $\alpha$ and $\beta$ be boolean combinations of atoms, pure-left $\Larrow$-formulas, pure-right $\Rarrow$-formulas, pure-past $\Uarrow$-formulas, and the formula $\Darrow(\varphi, \psi)$. Then, $\gamma \triangleq \Uarrow(\alpha, \beta)$ is equivalent to a separated formula where the only pure-future formula is $\Darrow(\varphi, \psi)$.
    \begin{note*}
        $\varphi$ and $\psi$ could be any formula in $\mathcal{X}_{until}$ that doesn't use the $\Uarrow$ and $\Darrow$ connectives. Simply  \cref{lemma:partial-separation-trees-step1} produces the formulas specified in the statement.
    \end{note*}
\end{lemma}
\begin{proof}
    We start by writing $\alpha$ and $\beta$ in their DNF and CNF forms respectively. This allows us to write $\gamma$ as
    \begin{equation*}
        \gamma \equiv \bigvee_i \bigwedge_j \Uarrow( \pm \alpha_{i, 1} \land \cdots \land \pm \alpha_{i, n_i}, \pm \beta_{j, 1} \lor \cdots \lor \pm \beta_{j, n_j})
    \end{equation*}
    where the $\alpha_{i, k_i}$ and $\beta_{j, k_j}$ can be atoms, instances of $\Darrow(\varphi, \psi)$, and syntactically pure $\pi$-formulas for $\pi \in \{\Larrow, \Rarrow, \Uarrow\}$. For each $\{i, j\}$, we can write the corresponding $\Uarrow$-formula in $\gamma$ as
    \begin{equation*}
        \gamma_{i, j} \triangleq \Uarrow(\alpha' \land \pm \Darrow(\varphi, \psi), \beta' \lor \pm \Darrow(\varphi, \psi))
    \end{equation*}
    where \textit{(naturally)} $\alpha'$ and $\beta'$ are conjunctions and disjunctions of syntactically pure past, left, present, and right formulas. As we've done many times before, we employ the eliminations detailed in \cref{sec:eliminations-trees} at each $\gamma_{i, j}$.

    Let $A = \{\alpha', \beta', \varphi, \psi\}$ and let $B = A \cup \{\theta\}$ for $\theta$ defined in \cref{eq:theta-defn}. From our discussion in \cref{sec:eliminations-trees}, we know that the eliminations produce syntactically-separated formulas that connect the items in $B$ using the boolean operators, the $\Uarrow$ connective, and the $\Darrow$ connective.

    By the nature of the formulas in $A$, any formula rooted by a $\Uarrow$ that takes atoms from $A$ and only uses the $\Uarrow$ connective is syntactically pure-past. Unfortunately, the addition of the formula $\theta$ doesn't maintain this property; it isn't separated, and its separated equivalent may contain a pure-future formula.

    Remember, $\theta$ only uses the $\Larrow$, $\Rarrow$ and $\Darrow$ connectives. We can produce a separated equivalent $\theta'$ using \cref{lemma:partial-separation-trees-step7}. To safely use the eliminations, we must argue that $\theta'$ contains no pure-future segment. We do so through the following claim.
    \begin{claim*}
        Take any two temporal structures $\mathcal{M} = (T, <, \prec, h)$ and $\mathcal{M}' = (T, <, \prec, h')$ and a point $t \in T$ such that $h$ and $h'$ agree on $t$, the left of $t$, and the right of $t$, but disagree on the future of $t$. We claim that
        \begin{equation*}
            \mathcal{M}, t \vDash \theta \longleftrightarrow \mathcal{M}', t \vDash \theta
        \end{equation*}
        \begin{note*}
            This claim implies that $\theta$ simply doesn't care about the future of $t$.
        \end{note*}
    \end{claim*}
    \begin{claimproof}
        We prove this by contradiction. Suppose, without loss of generality, that $\mathcal{M}, t \vDash \theta$ and $\mathcal{M}', t \nvDash \theta$. Further suppose that
        \begin{equation*}
            \begin{aligned}
                \mathcal{M}, t &\vDash \Larrow(\varphi \lor (\psi \land \Darrow(\varphi, \psi)), \top)\\
                \mathcal{M}', t &\nvDash \Larrow(\varphi \lor (\psi \land \Darrow(\varphi, \psi)), \top)
            \end{aligned}
        \end{equation*}
        The argument for the alternative is similar to this one.

        Suppose $\mathcal{M}, t \vDash \Larrow(\varphi, \top)$. Since $\varphi$ only uses the $\Larrow$ and $\Rarrow$ connectives, the truth of $\varphi$ at $t$ only depends on the assignments of atoms at $t$ and its siblings. Similarly, the truth of $\Larrow(\varphi, \top)$ at $t$ only depends on $t$ and its siblings. Since $\mathcal{M}$ and $\mathcal{M}'$ agree on these nodes, we have $\mathcal{M}', t \vDash \Larrow(\varphi, \top)$.

        Hence, we must have that $\mathcal{M}, t \vDash \Larrow(\psi \land \Darrow(\varphi, \psi), \top)$. Take the left sibling $s \in T$ such that $\mathcal{M}, s \vDash \psi \land \Darrow(\varphi, \psi)$. It's easy to see that $\mathcal{M}$ and $\mathcal{M}'$ agree on $s$, the siblings of $s$, and the future of $s$. Hence, since $\psi$ is only concerned with siblings and $\Darrow(\varphi, \psi)$ is pure-future, we must have $\mathcal{M}', s \vDash \psi \land \Darrow(\varphi, \psi)$. Hence, $\mathcal{M}', t \nvDash \Larrow(\varphi \lor (\psi \land \Darrow(\varphi, \psi)), \top)$, forming a contradiction.
    \end{claimproof}

    We now confidently replace all top-level \textit{(i.e., not under a connective)} pure-future $\Darrow$-formulas (if any exist) in $\theta'$ with $\bot$. This produces $\theta''$, a boolean combination of pure left, present, and right formulas. Let $A' = A \cup \{\theta''\}$. Observe that the desired property is now maintained, i.e., any pure $\Uarrow$ formula taking atoms from $A'$ is syntactically pure-past.

    Call the result of the eliminations $\gamma'_{i, j}$ for each $\{i, j\}$. Replace $\theta$ in $\gamma'_{i, j}$ with $\theta''$ to produce $\gamma''_{i, j}$. Observe now that $\gamma''_{i, j}$ is syntactically separated, and that its pure future segment consists of instances of $\Darrow(\varphi, \psi)$. Hence, the formula
    \begin{equation*}
        \gamma'' \triangleq \bigwedge_i \bigvee_j \gamma''_{i, j}
    \end{equation*}
    is separated and is equivalent to $\gamma$, proving the lemma.
\end{proof}
For the next step, we consider the possible nesting of $\Darrow(\varphi, \psi)$ inside multiple instances of the $\Uarrow$ connective.
\begin{lemma}
    \label{lemma:partial-separation-trees-step11}
    Let $\varphi$ and $\psi$ be boolean combinations of atoms, pure-$\Larrow$ formulas and pure-$\Rarrow$ formulas. Let $\gamma$ be a $\Uarrow$ formula such that the arguments of all $\Uarrow$ subformulas in $\gamma$ are boolean combinations of propositional atoms, pure-left $\Larrow$-formulas, pure-right $\Rarrow$-formulas, and the formula $\Darrow(\varphi, \psi)$. Then, $\gamma$ is equivalent to a separated formula where the only pure-future formula is $\Darrow(\varphi, \psi)$.
\end{lemma}
\begin{proof}
    In this proof, we induct on the maximum $\Uarrow$-depth of a $\Darrow(\varphi, \psi)$ in $\gamma$ that isn't under a $\Larrow$ or a $\Rarrow$.

    \begin{description}
        \item[Base case.] The maximum depth is 1. This is equivalent to \cref{lemma:partial-separation-trees-step10}.
        \item[Induction case.] Let the maximum depth be $n$, and $\Uarrow(\alpha_1, \beta_1), \ldots, \Uarrow(\alpha_m, \beta_m)$ be subformulas of $\gamma$ at $\Uarrow$-depth $n-1$ that contain $\Darrow(\varphi, \psi)$ as a subformula. The maximum depth of $n$ ensures that $\Darrow(\varphi, \psi)$ isn't further nested under a $\Uarrow$ in each $\Uarrow(\alpha_i, \beta_i)$.

        We now apply \cref{lemma:partial-separation-trees-step10} to each $\Uarrow(\alpha_i, \beta_i)$ to produce the separated equivalent $\gamma_i$, with the pure-future segment being $\Darrow(\varphi, \psi)$. Hence, replacing $\Uarrow(\alpha_i, \beta_i)$ with $\gamma_i$ produces a similar formula with a maximum depth of strictly less than $n$. Applying the induction hypothesis to this formula proves the lemma.
    \end{description}
\end{proof}
Note how the structure of the $\Uarrow$ subformulas are maintained in \cref{lemma:partial-separation-trees-step11}. The application of the elimination in \cref{lemma:partial-separation-trees-step10} produces $\Uarrow$-formulas over boolean combinations of pure-left, pure-right and similar $\Uarrow$ formulas. \cref{lemma:partial-separation-trees-step11} does't affect this.

We now consider the case of multiple $\Darrow$ formulas with the same restrictions on the arguments.
\begin{lemma}
    \label{lemma:partial-separation-trees-step12}
    Let $\varphi_1, \ldots, \varphi_n$ and $\psi_1, \ldots, \psi_n$ be any two sequences of $n$ $\mathcal{X}_{until}$ formulas that are boolean combinations of atoms, pure-$\Larrow$ formulas, and pure-$\Rarrow$ formulas. Let $\gamma$ be a $\Uarrow$-formula such that the arguments of every $\Uarrow$-subformula in $\gamma$ are boolean combinations of atoms, pure-left $\Larrow$-formulas, pure-right $\Rarrow$-formulas, pure-past $\Uarrow$-formulas, and formulas from $\{\Darrow(\varphi_i, \psi_i) \mid i \in \{1, \ldots, n\} \}$. Then, $\gamma$ is equivalent to a separated formula where the pure-future formulas are instances of $\{\Darrow(\varphi_1, \psi_1), \ldots, \Darrow(\varphi_n, \psi_n)\}$.
\end{lemma}
\begin{proof}
    Predictably, we induct on $n$.
    \begin{description}
        \item[Base case.] $n = 1$. This case is identical to \cref{lemma:partial-separation-trees-step11}.
        \item[Induction case.] Introduce atoms $r_1, \ldots r_{n-1}$. In $\gamma$, replace all occurrences of $\Darrow(\varphi_i, \psi_i)$ by $r_i$ to produce $\gamma'$. Separate $\gamma'$ according to \cref{lemma:partial-separation-trees-step11} to produce $\gamma''$. Replace the $r_i$ in $\gamma'$ with $\Darrow(\varphi_i, \psi_i)$ to produce $\gamma''$.

        Now, the $r_i$ can appear anywhere in $\gamma'$. If it only appears in pure-left, pure-present, and pure-right segments, replacing it with $\Darrow(\varphi_i, \psi_i)$ maintains the separated nature of the formula. Hence, we focus on the $\Uarrow$ formulas that contain the new atoms.

        Since \cref{lemma:partial-separation-trees-step11} doesn't affect the structure of the $\Uarrow$ formulas, their arguments must be boolean combinations of atoms, pure-left formulas, pure-right formulas, and smaller $\Uarrow$ formulas. If $r_i$ only appears inside these pure-left and pure-right subformula, the $\Uarrow$ formula remains pure-past. The only complex case is if the $r_i$ appears as in the pure-present segment as an atom.

        It's easy to see that we can apply the induction hypothesis on these $\Uarrow$ formulas, as they only contain $n-1$ different $\Uarrow$ formulas embedded in them. Applying them to each top-level $\Uarrow$ formula in $\gamma''$ proves this lemma.
    \end{description}
\end{proof}

We finally consider the case of the $\Darrow$ formula that contains $\Darrow$ subformulas. This lemma is \textit{slightly} more involved than the previous two.
\begin{lemma}
    \label{lemma:partial-separation-trees-step13}
    Let $\gamma_1, \ldots \gamma_n$ be any sequence of $n$ syntactically pure-future $\Darrow$ formulas, and let $\gamma$ be a $\Uarrow$ formula such that the arguments of every $\Uarrow$ subformula in $\gamma$ are boolean combinations of atoms, pure-left formulas, pure-right formulas, pure-past formulas, and formulas from $\{\gamma_1, \ldots, \gamma_n\}$. Then, $\gamma$ can be separated.
\end{lemma}
\begin{proof}
    We begin by defining a measure $\lambda$ on pure-future $\Darrow$ formulas.
    \begin{equation*}
        \lambda(\varphi) = \begin{cases}
            0 & \text{if $\varphi$ has no proper $\Darrow$ subformulas.}\\
            1 + \max_{\psi \in S(\varphi)} \left\{\lambda\left(\psi\right)\right\} & \text{if $S(\varphi)$ contains all proper $\Darrow$ subformulas in $\varphi$.}
        \end{cases}
    \end{equation*}
    We induct on the $(m, n)$, where $n$ is the length of the sequence and $m$ is the maximum $\lambda$-score of the formulas in $\{\gamma_1, \ldots, \gamma_n\}$.

    \begin{description}
        \item[Base case.] $m = 0$. All cases of $(0, n)$, for any $n$, is equivalent to \cref{lemma:partial-separation-trees-step12}.
        \item[Induction step.] For each $i$, denote the set of all proper $\Darrow$ subformulas of $\gamma_i$ as $A_i$.
        \begin{equation*}
            A_i = \{ \gamma_{i, 1}, \ldots, \gamma_{i, k_i} \}
        \end{equation*}
        where $k_i$ is the number of such subformulas. Introduce $k_i$ new atoms $\{r_{i, 1}, \ldots r_{i, k_i}\}$ and replace $\gamma_{i, j}$ in each instance of $\gamma_i$ in $\gamma$ with $r_{i, j}$. Call the resulting formula $\gamma'$.

        Observe that $\gamma'$ has introduces $\sum_{j = 1}^{n} k_j$ many new atoms, and that its $\lambda$ score is now 0. We can apply \cref{lemma:partial-separation-trees-step12} to $\gamma'$, producing $\gamma''$. At this stage, we replace all instances of $r_{i, j}$ with $\gamma_{i, j}$, producing $\gamma'''$.

        Again, if $r_{i, j}$ only appears in the pure-left, pure-right, and pure-present segments, $\gamma'''$ remains separated. Again, since \cref{lemma:partial-separation-trees-step12} doesn't change the structure of the $\Uarrow$ subformulas, their arguments in $\gamma''$ remain as boolean combinations of atoms, pure-left, pure-right, and similar $\Uarrow$ subformulas. If $r_{i, j}$ only appears in these pure-left or pure-right subformulas, $\gamma'''$ remains separated.

        Hence, the complex case involves $r_{i, j}$ appearing in a $\Uarrow$ formula without appearing under a $\Larrow$ or a $\Rarrow$. In the worst case, this will involve $\sum_{j=1}^{n} k_j$ new pure-future $\Darrow$ formulas. However, since $\lambda(\gamma_{i, j}) < \lambda(\gamma_i)$, we can apply the induction hypothesis on $\gamma'''$. This completes the proof.
    \end{description}
\end{proof}
At this point, our main result becomes a corollary.
\begin{corollary}[Partial Separation of $\mathcal{X}_{until}$]
    \label{corollary:partial-separation-final}
    Let $\gamma$ be a formula in $\mathcal{X}_{until}$ such that all $\Darrow$ subformulas of $\gamma$ are syntactically pure-future. Then, $\gamma$ can be separated.
\end{corollary}
\begin{proof}
    Let $\{\gamma_1, \ldots, \gamma_n\}$ be the set containing all top-level pure-future $\Darrow$ subformulas in $\gamma$. Introduce $n$ new atoms $r_1, \ldots, r_n$ and replace each instance of $\gamma_i$ in $\gamma$ with $r_i$. Call the resulting formula $\gamma'$.

    Separate $\gamma'$ according to \cref{corollary:partial-separation-without-down}, producing $\gamma''$. In a similar way, separate the arguments of all $\Uarrow$ subformulas in $\gamma'$. Now, replace all $r_i$ in $\gamma''$ with $\gamma_i$ to produce $\gamma'''$. Each top-level $\Larrow$, $\Rarrow$, and $\Darrow$ formulas remain pure after the substitution. Simply apply \cref{lemma:partial-separation-trees-step13} to the top-level $\Uarrow$ formulas to complete the proof.
\end{proof}

\subsection{Why some formulas can't be separated}
\label{sec:ef-games-start}

The reason we cannot extend the arguments in the previous section to separate mixed $\Darrow$ formulas is the cause of Marx's mistake: that the $\Darrow$ connective isn't deterministic. Unlike the leftward or ancestral paths, there are many downward paths starting from a node, and the $\Darrow$ can proceed down any of them.

Marx's proof in \cite{xpathComplete} fails because he expects to easily pull out $\Uarrow$ formulas from inside the scope of a $\Darrow$. This requires writing all $\Darrow$ formulas in the following form:
\begin{equation*}
    \Darrow(\alpha, \beta) \not\equiv \bigvee_i \bigwedge_j \Darrow( \pm \alpha_{i, 1} \land \cdots \land \pm \alpha_{i, n_i}, \pm \beta_{j, 1} \lor \cdots \lor \pm \beta_{j, n_j})
\end{equation*}
The argument presented in \cref{fig:darrow-a-b-and-c} shows why this isn't possible.

There are bigger consequences of this non-determinism. $\Darrow$ formulas that appear in the path condition of other $\Darrow$ formulas don't necessarily traverse the same path as their parent, as made evident in \cref{fnreliable}.

\begin{figure}[h]
    \centering
    \tikzfig{darrow-unreliable}
    \caption{\emph{The unreliability of $\Darrow$.} The present point satisfies $\Darrow(a, b \lor \Darrow(c, d))$. No point along the path to $a$ is a $c$.}
    \label{fnreliable}
\end{figure}

This is a problem because many eliminations in \cref{sec:eliminations-trees} require the ability to look for unfulfilled and dangerous points a little further down the main path. It is this \textit{unreliability} that we capitalize on in our arguments.

% WARNING: There's a super weird pdflatex bug here, where if I add a comment, or change the label of fnreliable, I risk pdflatex being unable to reference the figure properly. DO NOT SCREW WITH THIS HOUSE OF CARDS.

\textit{We must admit that this is a work in progress.} In this section, we will show a mixed $\Darrow$ formula that only uses the $\Darrow$ and $\Uarrow$ connectives that cannot be separated into a formula that only uses $\Darrow$ and $\Uarrow$. We believe it is perfectly possible to extend our argument to show that no separated formula that uses all the tools provided by $\mathcal{X}_{until}$ is equivalent to our mixed $\Darrow$ formula, but we don't present a proof for it here.

In the next few sections, we describe an EF game for Conditional XPath. We will use these games in our arguments.

\subsubsection{EF Games for $\mathcal{X}_{until}$}
In \cite{EtWi00}, Etessami and Wilke define EF-games for LTL formulas. Using these games, they show a strict hierarchy of expressive power that classifies formulas based on their \textsf{Until} ($U$) depths. In this subsection, we use their ideas to design EF-games for $\mathcal{X}_{until}$.

Naturally, our EF games are played on two structures $\mathcal{M}$ and $\mathcal{M}'$ by two players. We term the first player the spoiler and the second the duplicator. In our games, $\mathcal{M}$ and $\mathcal{M}'$ are labelled ordered trees where every node is labelled by a letter from a finite set $\Sigma$. This set is akin to the set $\mathcal{P}$ of all propositional atoms. At the beginning of the game, pebbles are placed at one chosen node in each structure. We refer to these nodes as the ``current'' nodes of the game. Typically, the chosen nodes are the roots of $\mathcal{M}$ and $\mathcal{M}'$.

At the start of each round, the spoiler begins by selecting a move type $\pi$ from the set of connectives $\{\Larrow, \Rarrow, \Uarrow, \Darrow\}$. The mechanism of each type is derived from the operation of its connective. All moves involve two stages of play. In the first stage, the spoiler picks one of the two structures, say $\mathcal{M}$, and places a second pebble at a node that can be reached from $\mathcal{M}$'s current node by moving in the direction of $\pi$. For instance, if $\pi$ is $\Larrow$, then the spoiler places a second pebble on a left-sibling of the current node. Importantly, if $\pi$ is $\Darrow$, then the spoiler can pick any descendant of the current node.

The duplicator responds by placing a second pebble in the other structure (which is $\mathcal{M}'$) that can be reached from its current node by moving in the same direction.

At this point, the spoiler can choose to end the round by removing the older pebbles from both structures. He can also choose to activate the second stage, which requires him to pick a node \emph{on the path connecting the two pebbles in the other structure} $\mathcal{M}'$. Note that the spoiler cannot pick the nodes that already have pebbles on them. He places a third pebble at his selected node. \cref{efgame-intro-fig} captures this possible play for $\pi = \Darrow$.

\begin{figure}[h]
    \centering
    \tikzfig{EF-game-intro}
    \caption{\emph{A $\Darrow$ move.} The spoiler started by picking the \textcolor{OliveGreen}{green} node in $\mathcal{M}$. The duplicator responded with the \textcolor{red}{red} node in $\mathcal{M}'$. The spoiler then picked the \textcolor{orange}{orange} node on the otherwise \textcolor{blue}{blue} path in $\mathcal{M}'$. The duplicator must respond by marking one of the \textcolor{violet}{violet} nodes.}
    \label{efgame-intro-fig}
\end{figure}

The duplicator responds by picking a node on the path connecting the two pebbles in $\mathcal{M}$, after which the older pebbles are removed and the round is complete.

The intuition behind the game is as follows. The spoiler hopes to use a $\pi$ formula that distinguishes the two structures. In the first stage, he picks a target point. If the duplicator fails to pick a similar target point, the spoiler doesn't need the second stage to win. However, if the duplicator succeeds, the spoiler must activate the second stage. The $\pi$ formula in his possession suggests that the duplicator's path differs from the spoiler's path in a detectable way. Accordingly, he picks a point on the duplicator's path that he believes to be different from every point on the spoilers path. The duplicator, naturally, attempts to refute the spoilers claim.

In our EF games, we limit the number of moves of a particular type. The sum of these limits forms the number of rounds of the game. Call an EF game a $(k_{\Larrow}, k_{\Rarrow}, k_{\Uarrow}, k_{\Darrow})$ game if the spoiler is allowed to make $k_\pi$ moves for each $\pi \in \{\Larrow, \Rarrow, \Uarrow, \Darrow\}$. For convenience, we refer to the tuple as a vector $\myvec{k} = (k_{\Larrow}, k_{\Rarrow}, k_{\Uarrow}, k_{\Darrow})$ (the order is important) and denote its sum $\sum k_\pi$ as $n$. Note that $n$ is the total number of rounds in the game.

The spoiler wins the $n$ round game if, after $n$ rounds, the current nodes at both structures are labelled differently. The duplicator wins otherwise. Note that, in all moves, the spoiler automatically wins if the duplicator is unable to find a node to place a pebble.

We now prove the usefulness of these games. We begin with the following definition.
\begin{definition}
    Let $\varphi$ be a $\mathcal{X}_{until}$ formula. For $\pi \in \{\Larrow, \Rarrow, \Uarrow, \Darrow\}$, the $\pi$ depth of $\varphi$ is simply the maximum number of $\pi$ connectives in a path from the root to a leaf in the formula tree of $\varphi$. Additionally, the connective depth of $\varphi$ is the maximum number of connectives in a path from the root to a leaf.
\end{definition}
Observe that if the $\pi$-depth of a formula $\varphi$ is $0$ \textit{iff} $\varphi$ doesn't use the connective $\pi$.

We now present the following essential lemma. Note that the set of atomic propositions $\mathcal{P}$ is now $\Sigma$, with the usual implications.
\begin{lemma}
    \label{lemma:count-limiter-efgames}
    Suppose the set of all atomic propositions $\mathcal{P}$ was finite. Then, up-to expressive equivalence, there are finitely many $\mathcal{X}_{until}$ formulas of $\pi$ depth $\leq k_\pi$ for each $\pi \in \{\Larrow, \Rarrow, \Uarrow, \Darrow\}$ and any tuple $\myvec{k} = (k_{\Larrow}, k_{\Rarrow}, k_{\Uarrow}, k_{\Darrow})$.
    \begin{remark*}
        For convenience, we say that these formulas are \emph{depth-bound} by $\myvec{k}$.
    \end{remark*}
\end{lemma}
\begin{proof}
    We show this by inducting on $n = \sum k_\pi$.
    \begin{description}
        \item[Base case.] $n = 0$. This is equivalent to showing that boolean operators can combine formulas from a finite set (in this case, $\mathcal{P}$) in finitely many unique ways. This can be observed by drawing a truth table.
        \item[Induction case.]
            For each $\sigma \in \{\Larrow, \Rarrow, \Uarrow, \Darrow\}$ such that $k_\sigma > 0$, let $\myvec{m_\sigma}$ be the vector defined as
            \begin{equation}
                \label{eq:m-sigma-defn}
                {(m_\sigma)}_\pi = \begin{cases}
                    k_\pi & \pi \neq \sigma \\
                    k_\pi - 1 & \pi = \sigma
                \end{cases}
            \end{equation}
            By the induction hypothesis, we can consider finite sets $A_\sigma$ of all formulas depth-bound by $\myvec{m_\sigma}$. For every pair of formulas $(\alpha, \beta)$ in $A_\sigma^2$, we can consider the formula $\sigma(\alpha, \beta)$. It's easy to see that each such $\sigma$ formula is depth-bound by $\myvec{k}$. Hence, take the set
            \begin{equation*}
                B_\sigma = \{\sigma(\alpha, \beta) \mid \alpha \in A_\sigma \land \beta \in A_\sigma \}
            \end{equation*}
            and define $C_\sigma \triangleq A_\sigma \cup B_\sigma$ for each $\sigma$. Similarly, define
            \begin{equation*}
                C \triangleq \bigcup_\sigma C_\sigma
            \end{equation*}
            for each $\sigma \in \{\Larrow, \Rarrow, \Uarrow, \Darrow\}$ such that $k_\sigma > 0$.

            Observe that $C$ is finite and contains all ``atomic'' formulas depth-bound by $\myvec{k}$. The fact that boolean operators can only combine formulas in $C$ in finitely many ways completes the proof.
    \end{description}
\end{proof}
Now for the main result of this subsection.
\begin{theorem}[Working principle of the EF games]
    \label{theorem:ef-games-working-principle}
    Let $\mathcal{M}$ and $\mathcal{M}'$ be two ordered trees with nodes labelled by alphabets from $\Sigma$, and let $t \in \mathcal{M}$ and $t' \in \mathcal{M}'$ be two points on these trees. Suppose a spoiler and a duplicator play a $\myvec{k} = (k_{\Larrow}, k_{\Rarrow}, k_{\Uarrow}, k_{\Darrow})$ round EF game on $\mathcal{M}$ and $\mathcal{M}'$ with the starting pebbles at $t$ and $t'$. Then,

    \begin{bracketenumerate}
        \item If all $\mathcal{X}_{until}$ formulas $\varphi$ depth-bound by $\myvec{k}$ cannot differentiate $\mathcal{M}, t$ from $\mathcal{M}', t'$, i.e.,
            \begin{equation*}
                \mathcal{M}, t \vDash \varphi \longleftrightarrow \mathcal{M}', t' \vDash \varphi
            \end{equation*}
            then, the duplicator has a winning strategy.

        \item If there exists some $\mathcal{X}_{until}$ formula $\varphi$ that is depth-bound by $\myvec{k}$ and
            \begin{equation*}
                \mathcal{M}, t \vDash \varphi \longleftrightarrow \mathcal{M}', t' \nvDash \varphi
            \end{equation*}
            then, the spoiler has a winning strategy.
    \end{bracketenumerate}
\end{theorem}
\begin{proof}
    We first show (1), and then (2).

    \proofsubparagraph{(1) No candidate formula differentiates the two structures.} In this case, we build the duplicator's winning strategy, by inducting on the number of rounds $n$.

    \begin{description}
        \item[Base case.] $n = 0$. Our assumptions ensure that the label of $t$ and $t'$ are identical, completing this case.
        \item[Induction step.]
            Without loss of generality, suppose the spoiler plays a $\sigma$ move and places a second pebble at the node $s$ in $\mathcal{M}$. Observe that this means $k_\sigma > 0$. Derive the tuple $\myvec{m_\sigma}$ from $\myvec{k}$ using \cref{eq:m-sigma-defn}. Build the set $A$ containing all (up-to equivalence) $\mathcal{X}_{until}$ formulas depth bound by $\myvec{m_\sigma}$ using \cref{lemma:count-limiter-efgames}.

            We now build the $\sigma$-formula that the spoiler hopes to use. Let $\alpha$ be the conjunction of all formulas $\varphi$ in $A$ such that $\mathcal{M}, s \vDash \varphi$. In a sense, $\alpha$ is the most \textit{specific} formula depth bound by $\myvec{m_\sigma}$ that the node $s$ can model. Hence, $\alpha$ must be the target condition.

            For the path condition, we need to produce the most specific formula that is true at all nodes on the path from $t$ to $s$. Let $R$ be the set of all nodes in $\mathcal{M}$ on this path. Naturally, $R$ is necessarily finite. For each node $r \in R$, let $\psi_r$ be the conjunction of all formulas $\gamma \in A$ such that $\mathcal{M}, r \vDash \gamma$. Clearly, $\psi_r$ is the most specific formula true at $r$. Finally, let $\beta$ be the disjunction of all $\psi_r$. This produces our path condition.

            It's easy to see that $\alpha$ and $\beta$ are well-defined formulas depth bound by $\myvec{m_\sigma}$. Hence, the formula $\sigma(\alpha, \beta)$ is well-defined and depth bound by $\myvec{k}$. Now note that, by construction,
            \begin{equation*}
                \mathcal{M}, t \vDash \sigma(\alpha, \beta)
            \end{equation*}
            This means that $\mathcal{M}', t' \vDash \sigma(\alpha, \beta)$. Thus, we are guaranteed a point $s' \in \mathcal{M}'$ with $\mathcal{M}', s' \vDash \alpha$ such that all nodes $r' \in \mathcal{M}'$ on the path from $t'$ to $s'$ must yield $\mathcal{M}', r' \vDash \beta$.

            The duplicator picks this point $s' \in \mathcal{M}'$ and places his pebble there. Suppose the spoiler decides to continue the game from $s$ and $s'$. Observe that the current game has $\myvec{m_\sigma}$ many moves left and that $\mathcal{M}, s$ and $\mathcal{M}', s'$ satisfy the same set of formulas depth-bound by $\myvec{m_\sigma}$. This allows us to employ the induction hypothesis to construct the remainder of the duplicator's strategy.

            Hence, suppose the spoiler picks a point $r' \in \mathcal{M}'$ on the path from $t'$ to $s'$. The duplicator is aware that $\mathcal{M}', r' \vDash \beta$. Since $\beta$ is a disjunction of $\psi_r$ formulas, $r'$ must model one of them. Let that formula be $\psi_u$ and its associated point be $u \in \mathcal{M}$. In other words, we will have a point $u$ on the path from $t$ to $s$ such that $\mathcal{M}', r' \vDash \psi_u$ and $\mathcal{M}, u \vDash \psi_u$.

            The duplicator places his pebble at $u$, completing the round. At this stage, we can similarly argue the validity of applying the induction hypothesis. This completes the proof of this case.
    \end{description}

    \proofsubparagraph{(2) A formula $\varphi$ differentiates the two structures.} Clearly, $\varphi$ is depth-bound by $\myvec{k}$. We similarly induct on $n$.

    \begin{description}
        \item[Base case.] $n = 0$. The assumptions force different labels on $t$ and $t'$, forcing a spoiler victory.
        \item[Induction step.]
            Let $A$ be the set of all $\{\Larrow, \Rarrow, \Uarrow, \Darrow\}$ subformulas that don't appear under a connective in $\varphi$. Clearly, if $\mathcal{M}, t \vDash \psi$ and $\mathcal{M}', t' \vDash \psi$ for every $\psi \in A$, both structures model $\varphi$. This means that there must be one subformula $\sigma(\alpha, \beta)$ for $\sigma \in \{\Larrow, \Rarrow, \Uarrow, \Darrow\}$ such that
            \begin{equation*}
                \mathcal{M}, t \vDash \sigma(\alpha, \beta) \qquad \text{and} \qquad \mathcal{M}', t' \nvDash \sigma(\alpha, \beta)
            \end{equation*}
            Since $\varphi$ is depth-bound by $\myvec{k}$, it follows that $\sigma(\alpha, \beta)$ is also depth-bound by $\myvec{k}$. Observe that this means $k_\sigma > 0$. Additionally, $\alpha$ and $\beta$ are depth bound by $\myvec{m_\sigma}$ (defined in \cref{eq:m-sigma-defn}), the sum of which is $< n$.

            Now, there is a point $s \in \mathcal{M}$ that can be reached by moving in the direction of $\sigma$ such that $\mathcal{M}, s \vDash \alpha$ and all points $r$ on the path from $t$ to $s$ yield $\mathcal{M}, r \vDash \beta$.

            The spoiler begins the round by choosing a $\sigma$ move and placing a pebble on $s$. If the duplicator picks a point $s' \in \mathcal{M}'$ such that $\mathcal{M}', s' \nvDash \alpha$, the spoiler continues the game from $s$ and $s'$. At this stage, we have a $\myvec{m_\sigma}$ game with $\alpha$ as the differentiator, letting us apply the induction hypothesis to produce the rest of the spoiler's strategy.

            Otherwise, by construction, there must be some point $r' \in \mathcal{M}'$ on the path from $t'$ to $s'$ such that $\mathcal{M}', r' \nvDash \beta$. The spoiler picks this point $r' \in \mathcal{M}'$. No matter which point $r$ the duplicator picks, we will have that $\mathcal{M}, r \vDash \beta$. We now have a $\myvec{m_\sigma}$ game with $\beta$ as the differentiator, allowing the use of the induction hypothesis at the game starting from $r$ and $r'$ to build the rest of the strategy, completing the proof.
    \end{description}
\end{proof}
We now present a neat consequence in a corollary.
\begin{corollary}
    \label{corollary:ef-games-working-principle}
    Let $\mathcal{M}$ and $\mathcal{M}'$ be two ordered trees with nodes labelled by alphabets from $\Sigma$, and let $t \in \mathcal{M}$ and $t' \in \mathcal{M}'$ be two points on these trees. Suppose a spoiler and a duplicator play a $\myvec{k} = (k_{\Larrow}, k_{\Rarrow}, k_{\Uarrow}, k_{\Darrow})$ round EF game on $\mathcal{M}$ and $\mathcal{M}'$ with the starting pebbles at $t$ and $t'$. Then,

    \begin{bracketenumerate}
        \item If the duplicator has a winning strategy, then all $\mathcal{X}_{until}$ formulas $\varphi$ depth-bound by $\myvec{k}$ cannot differentiate $\mathcal{M}, t$ from $\mathcal{M}', t'$, i.e.,
            \begin{equation*}
                \mathcal{M}, t \vDash \varphi \longleftrightarrow \mathcal{M}', t' \vDash \varphi
            \end{equation*}

        \item If the spoiler has a winning strategy, then there exists some $\mathcal{X}_{until}$ formula $\varphi$ that is depth-bound by $\myvec{k}$ and
            \begin{equation*}
                \mathcal{M}, t \vDash \varphi \longleftrightarrow \mathcal{M}', t' \nvDash \varphi
            \end{equation*}
    \end{bracketenumerate}
\end{corollary}
\begin{proof}
    Both (1) and (2) can easily be proved by contradiction using \Cref{theorem:ef-games-working-principle}. How? For (1), note that if there was a candidate formula $\varphi$, the spoiler would have a winning strategy, contradicting the fact that the duplicator has a winning strategy. % Make the Cref capital to make it work.
\end{proof}
This section shows how the EF games work. Notably, \Cref{theorem:ef-games-working-principle} still holds when $k_\sigma = 0$ for some $\sigma$. We use this fact in the next section.

\subsubsection{A game for separation}
\label{sec:game-separation}

In this section, we discuss games where the only move type is $\Darrow$.

Let's start by describing the structures that we plan to distinguish. We start by recursively constructing two classes of trees: $\mathcal{T} \triangleq \{T_i \mid i \in \mathbb{N}\}$ and $\mathcal{S} \triangleq \{S_i \mid i \in \mathbb{N} \}$. For the base case, define $T_0$ and $S_0$ to be the words $ba$ and $bba$ respectively. For all $i \geq 1$, the construction of $T_i$ begins with a downward path from its root (labelled $b$) to the root of an instance of $T_{i - 1}$ (which is also labelled $b$). This path has a single node, labelled $c$, between the two $b$s. At this point, copies of $T_{i - 1}$ and $S_{i - 1}$ are introduced as direct children of the root. $T_i$ is now fully formed.

The construction of $S_i$ is similar. We construct a downward path from the root (labelled $b$) to a second node (labelled $c$) and finally to a copy of $S_{i - 1}$, after which we introduce copies of $T_{i - 1}$ and $S_{i - 1}$ as children of the root.

For convenience, we construct two additional classes of trees: $\mathcal{T}' \triangleq \{T'_i \mid i \in \mathbb{N}\}$ and $\mathcal{S}' \triangleq \{S'_i \mid i \in \mathbb{N} \}$. Each $T'_i$ and $S'_i$ consist of a root labelled $c$ connected to copies of $T_i$ and $S_i$ respectively. We aim to play games on $T'_i$ and $S'_i$ for all $i \in \mathbb{N}$.

\begin{figure}[h]
    \centering
    \tikzfig{ef-structure-init}
    \caption{\emph{Recursive definition of $T_i$, $S_i$, $T'_i$, and $S'_i$.} Only the descendant relation is marked by arrows; the \textcolor{red}{red} arrows representing the sibling order are not shown.}
    \label{structureinit}
\end{figure}

The full recursive definition is pictorially represented in \Cref{structureinit}. Note that, since we don't need to accommodate $\Larrow$ and $\Rarrow$ moves, the ordering of the children of a node isn't important. We merely orient the root's children $T_{i - 1}$ and $S_{i - 1}$ to be at either side of the downward path containing a $c$ to simplify our discussion. We will refer to the downward path from the root to a leaf of any tree in $\mathcal{T}' \cup \mathcal{S}'$ that repeatedly takes the middle child at each branching point as the \textit{central path} of the tree.

The intuition behind this construction is as follows. The labelling of the central path of all trees in $\mathcal{T}'$ produces a word in $c (bc)^* ba$. Trees in $\mathcal{S}'$ cannot yield similarly labelled paths. In fact, all paths from the root to an $a$ in each $S' \in \mathcal{S}'$ must contain two consecutive $b$s. This indicates that, for all $i \in \mathbb{N}$,
\begin{equation}
    \label{eq:darrow-uarrow-game-eqn}
    T'_i, t'_i \vDash \Darrow(a, \lnot(b \land \Uarrow(b, \bot))) \quad \text{and} \quad S'_i, s'_i \nvDash \Darrow(a, \lnot(b \land \Uarrow(b, \bot)))
\end{equation}
Where $t'_i$ and $s'_i$ are the roots of $T'_i$ and $S'_i$ respectively. This gives us a mixed $\Darrow$ formula that differentiates $\mathcal{T}'$ and $\mathcal{S}'$.

We now claim that there is no pure $\Darrow$ formula that differentiates $\mathcal{T}'$ and $\mathcal{S}'$. Towards this, we present winning strategies for the duplicator in the $\Darrow$-only EF-game played on $T'_i$ and $S'_i$ that only affords the spoiler an insufficient number of $\Darrow$ moves. These strategies capitalize on the duplicator's ability to punish the spoiler for descending too quickly. As a consequence, the spoiler can take at most two steps forward in each round. We will elaborate on this in \cref{lemma:speedbreaker}. Notably, if the EF-game permitted a single additional $\Uarrow$ move, the application of \cref{theorem:ef-games-working-principle} to \cref{eq:darrow-uarrow-game-eqn} yields a winning strategy for the spoiler.

%We begin by formally proving the principle behind \cref{eq:darrow-uarrow-game-eqn}.
%\begin{lemma}
%    The following statements are true.
%    \begin{bracketenumerate}
%        \item All $T \in \mathcal{T}$ include a path from the root to a leaf of the form $(bc)^* ba$.
%        \item In all $S \in \mathcal{S}$, all paths from the root to a leaf must include two consecutive $b$s.
%    \end{bracketenumerate}
%\end{lemma}
%\begin{proof}
%    (1) requires simple induction. $T_0$ is simply the path $ba$, and all $T_i$ include a path labelled $bc$ to the root of $T_{i - 1}$, which by the induction hypothesis, contains a path labelled $(bc)^* ba$ from the root to a leaf.
%
%    Again, we show (2) by induction. $S_0$ has one candidate path to an $a$ labelled $bba$. The roots of all $S_i$ have three children: $S_{i - 1}$, $T_{i - 1}$, and a node labelled $c$. Paths to an $a$ that take either of the first two children will include two consecutive $b$s at the beginning. All remaining paths prepend a path to an $a$ from the root of a copy of $S_{i - 1}$ by a $bc$. The induction hypothesis implies the rest.
%\end{proof}
\begin{lemma}
    \label{lemma:speedbreaker}
    Suppose a spoiler and a duplicator play a $\Darrow$-only EF-game over $T'_n$ and $S'_n$ for some $n \geq 1$ where the starting pebbles are placed at the roots of $T'_n$ and $S'_n$. Then, a rational spoiler will not begin by placing a pebble inside any copy of $T_{n - 1}$ or $S_{n - 1}$ in either structure.
    \begin{remark*}
        This effectively limits the rate at which the spoiler descends the tree.
    \end{remark*}
\end{lemma}
\begin{figure}
    \centering
    \tikzfig{main-proof-start}
    \caption{\emph{The start of the game.} The starting positions are highlighted in \textcolor{blue}{blue}. The copies of $T_{n-1}$ and $S_{n-1}$ are marked in \textcolor{red}{red}.}
    \label{SpeedbreakerProofStartFig}
\end{figure}
\begin{proof}
    The starting configuration of the game is laid out in \Cref{SpeedbreakerProofStartFig}. Two different arguments are needed based on whether the connection point for the $T_{n - 1}$ or $S_{n - 1}$ to the trees is a $b$ or a $c$.

    \proofsubparagraph*{Case 1.}
    Without loss of generality, suppose the spoiler picks a node inside a copy of $T_{n - 1}$ that's connected to $T'_n$ at a $b$. The duplicator can respond with the corresponding node in the copy of $T_{n - 1}$ connected to $S'_n$ at a $b$. This response forces the equivalence of the futures of the newly marked nodes at $T'_n$ and $S'_n$.

    Hence, the spoiler must proceed onto the second stage of the move. He must now pick a node in $S'_n$ between the two pebbles. If he picks a node inside $T_{n - 1}$, the duplicator can respond with the corresponding node inside $T_{n-1}$ at $T'_n$. Again, this forces the equivalence of the future of the marked nodes, guaranteeing a duplicator victory.
    \begin{figure}[h]
        \centering
        \tikzfig{main-proof-step1}
        \caption{\emph{Case 1.} The spoiler picks the node shaded \textcolor{blue}{blue} inside $T_{n - 1}$. The duplicator responds by shading the corresponding node \textcolor{red}{red} in $S'_n$. The spoiler then shades a node \textcolor{violet}{violet} in $S'_n$, and the duplicator responds by shading the corresponding node \textcolor{orange}{orange} in $T'_n$. The optimal response of the spoiler in the second stage is to place a pebble at one of the \textcolor{OliveGreen}{green} $b$s.}
        \label{proofstep1fig}
    \end{figure}

    This means that, to avoid immediate defeat, the spoiler must pick a node that isn't inside $T_{n-1}$. The only possibility is the node $b$ that's next in the central path of $S'_n$. The duplicator mirrors his choice. \cref{proofstep1fig} describes this argument pictorially.

    Note that one can reach similar conclusions if the spoiler had picked a node inside the $T_{n - 1}$ attached to $S'_n$ at a $b$, or if he went for $S_{n - 1}$ instead of $T_{n - 1}$. We will not present the arguments for these possibilities here.

    \proofsubparagraph*{Case 2.}
    This case considers a spoiler in a hurry. Suppose, without loss of generality, he picks a node inside the copy of $T_{n - 1}$ attached to $T'_n$ at a $c$. If the duplicator were to pick the corresponding node inside the copy of $T_{n - 1}$ attached to $S'_n$ at a $c$, the spoiler would've succeeded. Unfortunately for him, the duplicator can choose the copy of $T_{n - 1}$ attached to a $b$ instead.

    Again, the duplicator's choice forces the spoiler onto the second stage of the move. He can only pick a node on the path between the pebbles in $S'_n$. Again, if he picks a node inside $T_{n - 1}$, the duplicator can mirror his choice in $T'_n$, guaranteeing his victory. The only node on the path outside $T_{n - 1}$ is, again, the $b$ that the duplicator's copy of $T_{n - 1}$ is connected to. Once the spoiler picks this node, the duplicator picks the corresponding $b$ in $T'_n$, completing the move, and slowing down the spoiler. \Cref{proofstep2fig} captures this case.

    Notice how the duplicator takes advantage of the structure of the game here. The path implied by the duplicators choice is ``smaller'' than the spoiler's choice. Accordingly, the $\Darrow$-path condition he satisfies is more exacting. This can be observed in the proof of \Cref{theorem:ef-games-working-principle} as fewer $\psi_b$ conditions; adding more $\psi_b$ formulas to the disjunction only makes the path more accommodating.
    \begin{figure}[h]
        \centering
        \tikzfig{main-proof-step2}
        \caption{\emph{Case 2.} The spoiler shades a node \textcolor{blue}{blue} inside $T_{n - 1}$ connected to $T'_n$ at a $c$. The duplicator responds by shading the corresponding node \textcolor{red}{red} in the copy of $T_{n - 1}$ attached to $S'_n$ at $b$, one level higher than $T'_n$. If the spoiler responds with the \textcolor{violet}{violet} node in $S'_{n}$, the duplicator can respond with the \textcolor{orange}{orange} node in $T'_{n}$. Hence, the spoiler must pick one of the \textcolor{OliveGreen}{green} $b$s.}
        \label{proofstep2fig}
    \end{figure}

    Again, note that this argument can be repurposed for the possible spoiler choice of picking a node inside the $S_{n - 1}$ connected to $S'_n$ at a $c$.\lipicsEnd

    These arguments show that any spoiler choice of a node in $T_{n - 1}$ or $S_{n - 1}$ forces him to move backwards to the $b$ directly underneath him at the start of the round. If the spoiler chose that $b$ initially, the duplicator will mirror his actions. Therefore, the choice of a node inside the $T_{n-1}$ and $S_{n - 1}$ is equivalent to the choice of a $b$.
\end{proof}
\cref{lemma:speedbreaker} makes it clear that the spoiler has two choices at the start of each round: move one step down to a $b$, or move two steps down to a $c$. Importantly, minor modifications to the proof yield similar results in similar games played between $T_n$ and $S_n$.
\begin{lemma}
    Suppose a spoiler and a duplicator play a $\Darrow$-only EF-game over $T_n$ and $S_n$ for some $n \geq 1$ where the starting pebbles are placed at the roots of $T_n$ and $S_n$. Then, a rational spoiler will not begin by placing a pebble inside any copy of $T_{n - 1}$ or $S_{n - 1}$ in either structure.
\end{lemma}
\begin{proof}
    Quite similar to the proof of \cref{lemma:speedbreaker}, and left to the reader. Note that this lemma implies that spoiler can only move the pebbles to the $c$ immediately below the root.
\end{proof}
We make use of these lemmas in the following theorem.
\begin{theorem}
    \label{theorem:working-of-darrow-game}
    Suppose a spoiler and a duplicator play a $(0, 0, 0, k)$ move EF game on $T'_n$ and $S'_n$ for any $n \in \mathbb{N}$. Then,
    \begin{bracketenumerate}
        \item If $k \leq n + 1$, the duplicator has a winning strategy.
        \item If $k = n + 2$, the spoiler has a winning strategy.
    \end{bracketenumerate}
\end{theorem}
\begin{proof}
    We first show (2) before (1).
    \proofsubparagraph*{(2).} We show this by recursively building a class of formulas $\Phi \triangleq \{ \varphi_i \mid i \in \mathbb{N}\}$ such that $\varphi_n$ has $\Darrow$-depth $n+2$ and can differentiate $T'_n$ and $S'_n$. Take
    \begin{equation*}
        \varphi_n \triangleq \begin{cases}
            \Darrow(b \land \Darrow(a, \bot), \bot) & n = 0\\
            \Darrow(c \land \varphi_{n - 1}, b) & \text{otherwise}
        \end{cases}
    \end{equation*}
    It's easy to verify the $\Darrow$ depth requirements. We now prove two properties of $\Phi$.
    \begin{claim*}
        For each $i \in \mathbb{N}$,
        \begin{alphaenumerate}
            \item $T'_i, t'_i \vDash \varphi_i$ and $S'_i, s'_i \nvDash \varphi_i$, where $t'_i$ and $s'_i$ are the roots of $T'_i$ and $S'_i$ respectively.
            \item For all $j \in \mathbb{N}$ such that $j < i$, we have $T'_j, t'_j \nvDash \varphi_i$ and $S'_j, s'_j \nvDash \varphi_i$.
        \end{alphaenumerate}
    \end{claim*}
    \begin{claimproof}
        We prove this by induction.
        \begin{description}
            \item[Base case.] $i = 0$. It's simple to see that
                \begin{equation*}
                    T'_0, t'_0 \vDash \Darrow(b \land \Darrow(a, \bot), \bot) \quad \text{and} \quad S'_0, s'_0 \nvDash \Darrow(b \land \Darrow(a, \bot), \bot)
                \end{equation*}
                This satisfies (a). For $i = 0$, it's easy so see that (b) is vacuously satisfied.
            \item[Induction step.] Suppose both conditions hold for all $i < n$. We will prove (b) before we prove (a).

                Take some $j < n$. In $T'_j$, there are many paths that satisfy $\Darrow(c, b)$. At the end of these paths lie the subtrees (in no order) $T'_{j-1}, T'_{j - 2}, S'_{j - 2}, T'_{j - 3}, \ldots$ and other smaller trees. These paths are shown in \cref{efFinalProofStep1} for $j = n$. By the induction hypothesis, for all $k < j$
                \begin{equation*}
                        T'_k, t'_k \nvDash \varphi_{n - 1}
                \end{equation*}
                Hence, $T'_j, t'_j \nvDash \Darrow(c \land \varphi_{n - 1}, b)$ which is equivalent to  $T'_j, t'_j \nvDash \varphi_n$. Slight modifications to this argument proves what we need for $S'_j$ as well, proving (b).
                \begin{figure}[h]
                    \centering
                    \tikzfig{ef-final-theorem-step1}
                    \caption{\emph{Paths satisfying $\Darrow(c, b)$.} The subtree rooted at the \textcolor{OliveGreen}{green} $c$ is $T'_{n - 1}$. The \textcolor{red}{red} $c$'s root the $T'_{n - 2}, S'_{n - 2}, T'_{n-3}, \ldots$.}
                    \label{efFinalProofStep1}
                \end{figure}

                Now, to show (a), observe that in $T'_n$, there is a path that satisfies $\Darrow(c, b)$ that ends in $T'_{n - 1}$, which satisfies $\varphi_{n-1}$. Hence,
                \begin{equation*}
                    T'_n, t'_n \vDash \Darrow(c \land \varphi_{n-1}, b) \implies T'_n, t'_n \vDash \varphi_n
                \end{equation*}
                And in $S'_n$, all paths that satisfy $\Darrow(c, b)$ end in $S'_{n - 1}, T'_{n -2}, S'_{n - 2}, \ldots$ and other smaller trees. Since none of them satisfy $\varphi_{n-1}$, we have
                \begin{equation*}
                    S'_n, s'_n \nvDash \Darrow(c \land \varphi_{n-1}, b) \implies S'_n, s'_n \nvDash \varphi_n
                \end{equation*}
                This completes the proof.
        \end{description}
    \end{claimproof}
    Since $\varphi_n$ differentiates $T'_n$ and $S'_n$ and has $\Darrow$-depth $n+2$, \cref{theorem:ef-games-working-principle} allows us to build a winning strategy for the spoiler in the $(0, 0, 0, n+2)$ round game over $T'_n$ and $S'_n$.

    \proofsubparagraph*{(1).} We show this by inducting on $n$. %As we proved in \cref{lemma:speedbreaker}, at each step, if the spoiler
    \begin{description}
        \item[Base case.] $n = 0$.
            Observe, through a simple case analysis, that no $\Darrow$ formula of $\Darrow$ depth 1 can distinguish $T'_0$ and $S'_0$. Applying \cref{theorem:ef-games-working-principle} gives us a winning duplicator strategy.
        \item[Induction case.]
            Now we play the game on $T'_n$ and $S'_n$ with the starting pebbles at $t'_n$ and $s'_n$, as shown in \cref{SpeedbreakerProofStartFig}. If the spoiler places a pebble inside $T_{n-1}$ or $S_{n-1}$, we apply the strategy detailed in \cref{lemma:speedbreaker} to ensure victory for the duplicator.

            To account for the other cases, if the spoiler moves down the central path one or two steps, the duplicator mirrors his actions by moving one or two steps down the central path of the other structure. It's easy to observe that he can always do this for $n > 0$.

            At this point, we can merge the strategy for $T'_{n-1}$ and $S'_{n-1}$ to complete the proof.
    \end{description}
\end{proof}
What has all this effort given us? Take the formula $\psi \triangleq \Darrow(a, \lnot(b \land \Uarrow(b, \bot)))$ capable of distinguishing $\mathcal{T}'$ and $\mathcal{S}'$. Suppose this formula is equivalent to a syntactically separated formula $\gamma$, which we can write as
\begin{equation*}
    \gamma \equiv \mathbb{B}(\gamma_{\Uarrow, 1}, \ldots, \gamma_{\Uarrow, n_{\Uarrow}}, \gamma_{p, 1}, \ldots, \gamma_{p, n_p} \ldots, \gamma_{\Darrow, 1}, \ldots, \gamma_{\Darrow, n_{\Darrow}}, \gamma_{\Larrow, 1}, \ldots, \gamma_{\Larrow, n_{\Larrow}}, \gamma_{\Rarrow, 1}, \ldots, \gamma_{\Rarrow, n_{\Rarrow}})
\end{equation*}
Suppose each pure-future component $\gamma_{\Darrow, i}$ only uses the $\Darrow$ formulas. Evaluating $\gamma$ at the root of some $T \in \mathcal{T}$ will set all pure-past, pure-left, and pure-right components $\gamma_{\Uarrow, i}$, $\gamma_{\Larrow, j}$ and $\gamma_{\Rarrow, k}$ to $\bot$. Replacing them with $\bot$ in $\gamma$, producing $\gamma'$, will not change the result. Hence, for any $n \in \mathbb{N}$,
\begin{equation*}
    T_n, t_n \vDash \gamma \Longleftrightarrow T_n, t_n \vDash \gamma'
\end{equation*}
Let the $\Darrow$ depth of $\gamma'$ be $k$. By \cref{theorem:working-of-darrow-game}, the duplicator has a winning strategy in the $k$-round $\Darrow$ game played over $T'_{k - 1}$ and $S'_{k - 1}$. This means, by \cref{corollary:ef-games-working-principle},
\begin{equation*}
    T_{k-1}, t_{k-1} \vDash \gamma' \Longleftrightarrow S_{k-1}, s_{k-1} \vDash \gamma'
\end{equation*}
But, as $\gamma \equiv \psi$,
\begin{equation*}
    T_{k-1}, t_{k-1} \vDash \gamma \Longleftrightarrow S_{k-1}, s_{k-1} \nvDash \gamma
\end{equation*}
Which is a contradiction!

Our hard work has allowed us to conclude that the $\gamma_{\Darrow, i}$ cannot only use the $\Darrow$ connective. If the mixed $\Darrow$ formula $\psi$ is separable, the pure-future components must contain a $\Rarrow$ or a $\Larrow$. This seems strange, given that $\psi$ is only concerned with a single downward path.

\subsection{A finer partition}
\label{sec:finer-partition}
% Note: I hoped to show that Marx's regions couldn't be made more granular as suggested by Gabbay in \cite{gabbay1994}. Unfortunately, the paper Gabbay based his proof on, \cite{Amir85}, uses a different approach to generalization. Amir's conditions are met by Marx's regions, diluting the importance of my findings somewhat.

One can partition the flow of time in many different ways. Certain partitions are more desirable than others. For example, while Gabbay's partitioning of the linear flow of time into past, present, and future yields interesting applications (see \cite{DecPastImpFuture89}), it isn't \textit{necessary} for implying expressive completeness. A more granular partition separating the flow into the present and everything \textit{but} the present yields \cref{theorem:separation-theorem-linear-time} just as well. But, it's difficult to show a separation property over this granular partition without separating the past from the future.

In his generalization of \cref{theorem:separation-theorem-linear-time} in \cite{gabbay1994}, Gabbay was partial to partitions that produced \textit{invertible} regions. These partitions produced regions such that all points $s$ in a region $R$ with respect the present point $t$ viewed $t$ as belonging to the same unique region $R'$. Gabbay's partition of linear time satisfies this property: all points in the past of $t$ view $t$ as a future point, and all points in the future of $t$ view $t$ as a past point.

It's easy to observe that this property doesn't hold for Marx's regions. Points in the left of $t$ can view $t$ either as a right point or a past point.  More egregious is the past of $t$: based on the node, $t$ can be a future point, a left point, a right point, or a past point.

This motivates the following finer partitioning of ordered trees that produces invertible regions.
\begin{itemize}
    \item The present region $\{t\}$, where $t$ is the present point.
    \item The future region $\{s \mid t < s\}$, reachable by $\Darrow$
    \item The ancestors $\{s \mid s < t\}$, reachable using $\Uarrow$.
    \item The left-sibling regions $\{s \mid s \prec t\}$, requiring $\Larrow$.
    \item The right-sibling regions $\{s \mid t \prec s\}$, needing $\Rarrow$.
    \item The future of the left $\{s \mid \exists r.\, r \prec t \land r < s\}$, reachable through a $\Darrow$ after a $\Larrow$.
    \item The future of the right $\{s \mid \exists r.\, t \prec r \land r < s\}$, similarly reachable by a $\Darrow$ inside a $\Rarrow$.
    \item The left siblings of the past $\{s \mid \exists r.\, r < t \land s \prec r\}$, reachable by $\Uarrow(\Larrow(\cdots))$.
    \item The right siblings of the past $\{s \mid \exists r.\, r < t \land r \prec s\}$, reachable by $\Uarrow(\Rarrow(\cdots))$.
    \item The future of the left siblings of the past $\{s \mid \exists r.\, (r < t) \land (\exists u.\, u \prec r \land u < s)\}$, requiring $\Uarrow(\Larrow(\Darrow(\cdots)))$.
    \item The future of the right siblings of the past $\{s \mid \exists r.\, (r < t) \land (\exists u.\, r \prec u \land u < s)\}$, requiring $\Uarrow(\Rarrow(\Darrow(\cdots)))$.
\end{itemize}
Here, $<$ is the descendant relation and $\prec$ denotes the sibling order. These eleven regions are pictorially represented in \cref{finerSeparationFig}. It's easy to verify that this partition produces invertible regions. For instance, all points in the future of $t$ view $t$ as an ancestor, all points in the future of the left of $t$ view $t$ as the right of the past, and so on.
\begin{figure}[h]
    \centering
    \tikzfig{separation-tree-finer}
    \caption{\emph{A finer separation of ordered trees.} The eleven different colours depict the eleven regions.}
    \label{finerSeparationFig}
\end{figure}

Unfortunately, it is not possible to separate all $\mathcal{X}_{until}$ formulas into these regions. Intuitively, this is because one may not be able to control the entry to a region using connectives meant for that region alone. For example, the specific point of entry to the future-of-the-left in a formula of the form $\Larrow(\alpha \land \Darrow(\cdots))$ is determined by the points on the left that satisfy $\alpha$. We capitalize on this intuition in the following lemma.
\begin{lemma}
    The pure-left formula $\gamma \triangleq \Larrow(a \land \Darrow(a, \bot), \lnot a)$ cannot be separated into a boolean combination of left-sibling formulas and future-of-the-left formulas.
\end{lemma}
\begin{proof}
    We prove this by contradiction. Suppose $\gamma$ can be separated into boolean combinations of left-sibling and future-of-the-left formulas. Let the separated equivalent be $\gamma'$, defined as
    \begin{equation*}
        \gamma' \triangleq \mathbb{B}(\alpha_1, \ldots, \alpha_n, \beta_1, \ldots, \beta_m)
    \end{equation*}
    Where the $\alpha_i$ are pure left-sibling formulas and the $\beta_j$ are pure future-of-the-left.

    Define a class of trees $\mathcal{T} = \{ T_i \mid i \in \{1, 2, \ldots, 2^n + 1\}\}$ as follows. Each $T \in \mathcal{T}$ a root with $2^{n} + 2$ children. We will evaluate formulas at the rightmost child of each tree.

    Each child of the root is either a leaf which doesn't model the atom $a$, or a node that models $a$ and has a single child that also models $a$. Importantly, for each $T \in \mathcal{T}$, only one child of the root models $a$. The trees in $\mathcal{T}$ differ at the positioning of this unique child.

    For $i \in \{1, 2, \ldots 2^n + 1\}$, the tree $T_i \in \mathcal{T}$ positions the child that models $a$ at $i$ $\Larrow$ steps away from the rightmost child (i.e., the present point). This is demonstrated in \cref{fineRegionsTree}.
    \begin{figure}[h]
        \centering
        \tikzfig{why-finer-regions-wont-work}
        \caption{\emph{The tree $T_i$.} The present node is shaded black. The \textcolor{violet}{violet} nodes model $a$ and the \textcolor{blue}{blue} ones do not. Note that the \textcolor{violet}{violet} sibling is $i$ steps away from the present.}
        \label{fineRegionsTree}
    \end{figure}

    It's easy to see that, for all $T \in \mathcal{T}$, $T, t \vDash \gamma$, where $t$ is the rightmost child of the root of $T$. We will now build a tree $S$ that models $\gamma'$ but not $\gamma$.

    Take the $n$ pure-left components of $\gamma'$. For $i \in \{1, \ldots, n\}$, each $\alpha_i$ is either true or false at each $T \in \mathcal{T}$. For simplicity, denote the operation $[\;]_{T, t}$ as
    \begin{equation*}
        [\alpha_i]_{T, t} = \begin{cases}
            1 & T, t \vDash \alpha_i\\
            0 & T, t \nvDash \alpha_i
        \end{cases}
    \end{equation*}
    There are $2^n$ possible configurations of $([\alpha_1]_{T, t}, \ldots, [\alpha_n]_{T, t})$, and because there are $2^n+1$ formulas in $\mathcal{T}$, for two trees $T_i, t_i$ and $T_j, t_j$, we have
    \begin{equation*}
        ([\alpha_1]_{T_i, t_i}, \ldots, [\alpha_n]_{T_i, t_i}) = ([\alpha_1]_{T_j, t_j}, \ldots, [\alpha_n]_{T_j, t_j})
    \end{equation*}
    We build $S$ using $T_i$ and $T_j$. Without loss of generality, let $i < j$. We again begin with a root that has $2^n + 2$ children, and evaluate formulas at the rightmost child. For all $k \in \{1, \ldots, 2^n + 1\}$ with $k \neq i$, the node $k$ left-steps away from the rightmost node doesn't model $a$. The node $i$ steps away \textit{does} model $a$, but has no children. Instead, the node $j$ steps away has a child that models $a$. This tree is depicted in \cref{fineFailureTree}.
    \begin{figure}[h]
        \centering
        \tikzfig{why-finer-part2}
        \caption{\emph{The tree $S$.} The present node is shaded black. The nodes modelling $\lnot a$ are shaded \textcolor{blue}{blue}, and the node modelling $a$ is shaded \textcolor{violet}{violet}. The present node has $2^n + 1$ left-siblings.}
        \label{fineFailureTree}
    \end{figure}

    Notice that $S$ agrees with $T_i$ on the left-sibling region, and agrees with $T_j$ on the future-of-the-left. Hence, for all $k_\alpha \in \{1, \ldots, n\}$ and $k_\beta \in \{1, \ldots, m\}$,
    \begin{equation*}
        S, s \vDash \alpha_{k_\alpha} \Longleftrightarrow T_i, t_i \vDash \alpha_{k_\alpha}\\
    \end{equation*}
    But, since $[\alpha_k]_{T_i, t_i} = [\alpha_k]_{T_j, t_j}$, we must have
    \begin{equation*}
        S, s \vDash \alpha_{k_\alpha} \Longleftrightarrow T_j, t_j \vDash \alpha_{k_\alpha}\\
    \end{equation*}
    And since $S$ agrees with $T_j$ on the future-of-the-left,
    \begin{equation*}
        S, s \vDash \beta_{k_\beta} \Longleftrightarrow T_j, t_j \vDash \beta_{k_\beta}\\
    \end{equation*}
    Giving us
    \begin{equation*}
        S, s \vDash \gamma' \Longleftrightarrow T_j, t_j \vDash \gamma'
    \end{equation*}
    But, it's easy to see that
    \begin{equation*}
        S, s \nvDash \gamma
    \end{equation*}
    forcing a contradiction and proving the lemma.
\end{proof}
Hence, in order to achieve separation, the left-siblings and the future-of-the-left must be merged. Similar arguments can be used to show that the right-siblings and the future-of-the-right must also be merged. We can continue these arguments and merge the ancestors, their left-siblings, their right-siblings, the future of their left-siblings, and the future of their right siblings, deriving Marx's partitioning scheme.

The author believes that this result provides a small amount of justification for Marx's choices.

\section{Conclusions and Future Work}
\label{sec:conclusions}
In this thesis, we studied Marx's proposed separation of $\mathcal{X}_{until}$, a temporal logic that forms the core of Conditional XPath. We have extended Marx's arguments to their natural conclusion: a partial separation property over a subset of $\mathcal{X}_{until}$ formulas that only contain pure-future $\Darrow$ subformulas. We then showed how certain simple mixed $\Darrow$ formulas cannot be separated using $\Darrow$ and $\Uarrow$ alone. Separately, we also justified Marx's partitioning of ordered, unranked trees by showing how a finer, more desirable partition wouldn't work.

One definite area for future work is to extend the results in \cref{sec:game-separation} to games that allow $\Larrow$ and $\Rarrow$ moves. If such a result can be established (which the author believes is possible), the separation property proposed by Marx in \cite{xpathComplete} will finally be proven wrong. The author was able to procure two very messy ordered \textit{infinite} trees that couldn't be distinguished by $\Darrow$, $\Larrow$ and $\Rarrow$ moves but could easily be distinguished using a constant number of $\Uarrow$ and $\Darrow$ moves, but the author believes a cleaner result is possible.

Regardless of whether we are able to disprove Marx's separation property, we would like to have \textit{some} separation of $\mathcal{X}_{until}$. Experimenting with new connectives and other finer partitions with the goal of finding the right combination of expressiveness and separation is another direction of future work.

Finally, one could also explore applications of separation of ordered, unranked trees. Inspired applications like \cite{DecPastImpFuture89} motivate newer and more interesting ways to use temporal logics, and similar applications over tree structures are naturally desirable.

%%
%% Bibliography
%%

%% Please use bibtex,

\bibliography{refs}

% \appendix
\end{document}
